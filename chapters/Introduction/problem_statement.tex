All these read/write operations to the database do not come without
a cost. First and foremost is the network latency. Even with high
throughput, low latency network between the RM/RT and the database, still
it takes more time than in-memory operations. Especially in cases
where RM operations need more than one round-trip to the database, the
difference in performance is noticeable.

A great advantage of using a relational database is the support of
foreign key constraints. The information that we store is semantically
related. In a SQL schema that is directly translated to foreign key
constraints. A trivial example is information regarding the containers
running on node and information regarding the node itself. Clearly, we
should never run into a situation where we try to persist a container
running on a node without having information about the node at
all. Moreover, foreign key constraints work the other way around. By
removing the information about a node from the database we want the
information about running containers on that node to be purged
too. Such constraints, although they seem tempting to use, pose a
great performance degradation. Particularly when we aim for millions
of operations per second, the use of foreign key constraints should be
limited and very well designed.

The transaction state manager of Hops should try to commit
various states in parallel as much as possible but on the other hand
ensure the order of two colliding states. For example there should
never be the case where two states change information about the same
node in the cluster and yet be committed in parallel. Similarly, two
states that change information in the database about different
entities should be committed in parallel.

The distributed MySQL cluster database is a central building block in our
architecture. A slow commit and process time will result in less events being
streamed and handled by the RM, directly affecting the cluster
utilization and the view of the cluster from the RM perspective.
A slow commit time will decrease the
rate of handled events while increase the rate of queued events that
are waiting to be handled. Our goal is to be able to scale up to 10000 NMs
with multiple RTs. With the current mechanism this is not possible due
to delays both in the transaction manager mechanism and in the commit
time.