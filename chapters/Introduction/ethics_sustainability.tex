In a wide range of academic areas such as sociology, economics,
medical sciences, etc the ability to effectively process and gain
valuable insights from a big amount of data with simpler algorithms,
trump other more sophisticated models with less data
\cite{10.1109/MIS.2009.36}. Researchers gather a huge volume of data
from different sources. Providing tools to process and analyze these
data is crucial for understanding the human behaviour and validate
economic theories or more important theories on medical treatments.

In order to process these data a cluster of computers should be put in
place. The goal of this thesis is to increase the number of resources
that are allocated at any time in a cluster. This will increase both the
number of jobs that researchers can issue to analyze their data and
the resource requirements for their processing tasks. Earlier we have
given the example of DNA sequencing that generates a lot of data. In
cases where medical data are involved, certain procedures should be
followed to provide privacy. Although this work does not focus on
security, it is of great importance to isolate and fence properly the
data stored and filter who can access them. Also, in most of the
cases cluster resources are shared among different units in the same
organization. Even different competitive organizations might share resources on
the cloud. The resource management system should enforce rules on the
tenants of the cluster regarding the fair share of the resources and
prevent situations where one tenant has allocated resources even
though they are not used.

Every big organization nowadays has its own cluster to perform big
data analytics. Smaller organizations and companies that cannot afford
provision their own cluster turn to cloud service providers. In both
cases, computers are operating 24 hours a day, every day consuming
a considerable amount of energy. Having a cluster of computers
underutilized means that they consume energy while being idle. By
increasing the cluster utilization, computers will consume energy for
performing more tasks rather than being idle. For the same reason, an
organization could profit. In the case where they operate their own
cluster, the money paid for energy consumption and hardware
provisioning are ``invested'' for performing tasks and not being
idle. In the case where they use a cloud service provider, then by
utilizing more resources the organizations could lease them (the
resources) for less time, thus paying less money.