The impact of the aggregation mechanism in the Transaction Manager was
measured with a synthetic payload that should be persisted in
NDB. Then we measured the total number of transaction commits that
took place, the average time required by each transaction and the
total time taken for the payload to be committed. The comparison is
between the \emph{Simple} and \emph{Aggregated} commit mechanism.

The payload was for creating 100 Applications. Each application would
uniform randomly request or release 1000 containers on 30 nodes
chosen also uniformly at random. A new Transaction State object object
was created with 60$\%$ probability over re-using an existing
one. Finally, when all this information was tracked by the Transaction
State objects were persisted in the database in a foreign-key free
schema. The reason to pick hosts
and TS objects randomly is to create conflicts in the commit mechanism
of the Transaction Manager that would block TSs. In that way we could
measure the impact of the aggregation mechanism.

\begin{table}
\centering
\begin{tabular}{| c | c | c | c |}
\hline
  & Simple & Aggregated \\
\hline
\# commits & 9867 & \textbf{79} \\
\hline
AVG time/TS commit (ms) & \textbf{10.4} & 74.1 \\
\hline
Time for all TXs (ms) & 21762 & \textbf{8088} \\
\hline
\end{tabular}
\caption{Comparison of commit mechanisms}
\label{tab:ev_commit_mechanism}
\end{table}

The results of the evaluation are presented in Table
\ref{tab:ev_commit_mechanism}. It is clear that the aggregation
mechanism minimized the commits in the database. In the Simple mechanism
each TS was performing a commit phase whereas in the Aggregated mechanism,
multiple TS are squashed into a single commit. It is normal that the
average time needed to commit a single TS in the database is greater
in the new version, since now the TS contains much more
information. In some cases the AggregatedTransactionState was
aggregating up to 200 Transaction States. Also, some times the commit
time per TS was above 150 but the standard deviation is very low with
most of the results near the mean value. Overall, we spent more
than the half of the time we spent with the Simple mechanism to persist
the whole payload.

*** Run simulation simple/aggregate, async, nocache and report cluster
utilization and wait in the queue time