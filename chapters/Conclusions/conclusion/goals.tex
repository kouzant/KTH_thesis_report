In Chapter \ref{chap:introduction} the goals of this project were
set. The primary goals was to improve the cluster utilization and the
number of heartbeats processed by the scheduler. In order to achieve
those goals we have also set some sub-goals. With the solutions proposed
in Chapter \ref{chap:implementation} all the sub-goals were met. In
particular, with the removal of the foreign key constraints and the
DTO caching mechanism the transactional commit time was decreased
dramatically. Some sort of asynchronous API was provided by the
garbage collector service. It is provided only for a small sub-set but
still it made big impact to the performance of the system. The new
aggregation mechanism of the transaction manager of Hops-YARN helped
the blocked transactions to be committed faster which in turn improved the
parallelization of the system. Finally, in each step of the
implementation an evaluation was done to prove the performance impact
and guide us to new bottlenecks.

Since all the sub-goals were met it was expected to achieve the
primary goals. As Chapter \ref{chap:evaluation} indicates the two
primary goals were also accomplished. Both cluster utilization and the
heartbeat ratio was improved.
