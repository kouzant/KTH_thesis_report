In Hops-YARN, as it is already mentioned, MySQL Cluster NDB is used
both as a communication transport and as a persistent storage for
recovery. At the beginning of the project a thorough profiling of the
execution workflow has been done to identify the bottlenecks of the
system. The first step was to remove the foreign key constraints from
the database schema used by Hops-YARN. In Section
\ref{sec:fk_constraints} is explained how they were replaced by
application logic that performs primary key operations. Section
\ref{sec:tx_aggregation} discusses the evolution of the commit
mechanism which squashes several blocked transaction into a ``big''
one reducing the number of commits in the back-end database
system. Section \ref{sec:gc_service} presents the Garbage Collector
service of Hops-YARN that removes asynchronously old values from the
database. With that solution the commit time dropped more improving
the overall performance of the system. Finally, Section
\ref{sec:dto_caching} explains how the shortcoming of ClusterJ for
creating DTOs was bypassed by having created them ahead of time in a
per session cache.

After a detailed explanation of the solutions proposed, in Chapter
\ref{chap:evaluation} follows the evaluation. Each solution is
evaluated separately by simulating real world traces. In each case key
characteristics are examined and how they have been improved. In
Section \ref{sec:performance_overview} there is an overall performance
overview in terms of cluster utilization and heartbeats processed by
the scheduler. The comparison is made among the version of Hops-YARN
before this project, the final version with all the modification
proposed and the upstream Apache YARN. The figures show that there was
a clear improvement, in both evaluation parameters, when the two
version of Hops-YARN are compared. Finally, as far as the cluster
utilization is concerned, the performance is comparable with Apache
YARN in clusters with thousands of nodes.