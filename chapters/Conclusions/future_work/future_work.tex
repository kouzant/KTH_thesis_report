A few things have been left undone due to time limitation and are
discussed in this section for future work.

In most of the cases the evaluation has been done using simulations
measuring among others the cluster utilization while in others the
evaluation has been done using benchmarks. It would be more complete
if in all cases the benchmarks were supported by simulation
results. That way the perfomance improvements introduced by each step
would be more clear.

Currently we do not have any insight on the content of Transaction
State objects, thus we treat them equally. In real world scenarios,
during the allocation of resources, the Transaction State would carry
more information about allocated containers than when the cluster is
full and no further allocations can be made. A fine-grained inspection
on the content of a TS might improve performance further more. For
example in the commit mechanism, when an Aggregated Transaction State
object is overloading NDB, the transaction will roll-back and the
aggregation policy will enforce a lower limit. If we had any
information on the content of the TS before hand, we could have
avoided that situation.

As it is already mentioned, the Garbage Collector service does not
perform primary key operations. It was a design decision that would
not complicate the mechanism and burden the commit time by committing
more information. It is worth trying to persist all the columns needed
for the primary keys to be reconstructed and measure the
performance. For sure the deletion time would be lower but it
remains to be proven if that will have any impact on the performance
of Hops-YARN in general.

Last but not least, the batching system explained in Section
\ref{ssec:impl_batch_system} should be improved. Changing the number
of RPCs that are batched together did not change the
performance. Every RPC arriving in the ResourceManager among others it
should get the current Transaction State object from the Transaction State
Manager. This operation could take $0.06$ ms. If we consider a cluster
with 10000 NodeManagers heartbeating every second, it is needed 600 ms
just to acquire the object. At the time of writing that was the latest
bottleneck encountered and for sure it is worth for future investigation.