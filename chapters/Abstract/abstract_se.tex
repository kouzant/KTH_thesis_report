 Nu för tiden lagrar stora organisationer och forskningsinstitutioner enorma mängder data.
För att kunna utvinna någon värdefull information från dessa data behöver den bearbetas
av ett kluster av datorer. När flera datorer gemensamt ska bearbeta data behöver de utgå
från ett så kallat ``distributed processing framework''. I dagsläget är Apache Hadoop det
mest använda ramverket för distribuerad lagring och behandling av data. Detta examensarbete
är har genomförts vid SICS Swedish ICT där vi byggt Hops, en ny distribution av
Apache Hadoop som drivs av ett distribuerat MySQL Cluster NDB som erbjuder en hög tillgänglighet.
Hops-YARN är Hops ramverk för resurshantering med distribuerade ResourceManagers som lastbalanserar
deras ResourceTrackerService. I detta examensarbete använder vi Hops-Yarn på ett sätt där ``back-end''
databasen flitigt används för att hantera ResourceManagerns metadata och inkommande RPC-anrop. Vår
konfiguration erbjuder en hög feltolerans och återställer sig mycket snabbt vid
felberäkningar. Vidare används NDB-klustrets Event API för att ResourceManager ska kunna
kommunicera med den distribuerade ResourceTrackers.

Detta projekt syftar till att optimera de mekanismer som används för ihållande metadata
i NDB både i termer av transaktions begå tid men också i termer av pre-
bearbeta dem medan samtidigt garantera enhetlighet i RM: s tillstånd. ResourceManagerns tillstånd
i RAM-minnet får under inga omständigheter
avvika från det tillstånd som finns lagrat i NDB:n. Med dessa mål i åtanke undersöktes flera
lösningar som förbättrar prestandan och därmed gör Hops-Yarn jämförbart med Apache YARN.
De lösningar som föreslås i denna uppsats förbättrar “pure commit time” när en transaktion
görs i ett MySQL Cluster samt förbehandlingen och parallelismen i vår Transaction Manager.
Resultaten tyder på att Hops prestanda ökade dramatiskt vilket ledde till ett effektivare
nyttjande av tillgängliga resurser i ett kluster bestående av ett tusental datorer. När
nyttjandet av tillgänliga resurser i ett kluster förbättras med några få procent kan
organisationer spara mycket pengar.

