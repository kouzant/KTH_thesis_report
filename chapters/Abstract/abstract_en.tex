Large organizations and research institutes store a huge volume of data nowadays.
In order to gain any valuable insights distributed processing frameworks over a
cluster of computers are needed. Apache Hadoop is the prominent framework for distributed
storage and data processing. At SICS Swedish ICT we are building Hops, a new distribution
of Apache Hadoop relying on a distributed, highly available MySQL
Cluster NDB to improve performance. Hops-YARN is the resource management framework of Hops
which introduces distributed resource management, load balancing the
tracking of resources in a cluster. In Hops-YARN we make heavy usage of the
back-end database storing all the resource manager metadata and
incoming RPCs to provide high fault tolerance and very short recovery
time.

This project aims in optimizing the mechanisms used for persisting
metadata in NDB both in terms of transactional commit time but also
in terms of pre-processing them. Under no condition should the in-memory RM
state diverge from the state stored in NDB. With these goals in mind
several solutions were examined that improved the performance of the
system, making Hops-YARN comparable to Apache YARN with the extra benefits
of high-fault tolerance and short recovery time. The solutions
proposed in this thesis project enhance the pure commit time of a
transaction to the MySQL Cluster and the pre-processing and parallelism
of our Transaction Manager. The results indicate that the performance
of Hops increased dramatically, utilizing more resources on a cluster
with thousands of machines. Increasing the cluster utilization by a
few percentages can save organizations a big amount of money.
