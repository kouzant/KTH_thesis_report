%%%%%%%%%%%%%%%%%%%%%%%%%%%%%%%%%%%%%%%%%%%%%%%%%%%%%%%%%%%%%%%%%%%%%%%
%% $Id: report.tex,v 1.5 2005/02/09 21:06:42 lindstrm Exp $
%%%%%%%%%%%%%%%%%%%%%%%%%%%%%%%%%%%%%%%%%%%%%%%%%%%%%%%%%%%%%%%%%%%%%%%
%% costhesis usage example
%% modified and added to by GQMJr
%%%%%%%%%%%%%%%%%%%%%%%%%%%%%%%%%%%%%%%%%%%%%%%%%%%%%%%%%%%%%%%%%%%%%%%
%
% The costhesis package accepts the following options
%
%   Document types:
%     msc               - Master Thesis
%     bsc		- Kandidate Thesis
%
%   Layout options:
%
%   Other options:
%     blank             - Removes pagenumbers and headers from empty pages
%     blankmsg          - Prints a message of intent on empty pages
%     scheader          - Typeset headers in SMALL CAPS shape (default)
%     slheader          - Typeset headers in slanted shape 
%
%
%
%

\documentclass[12pt,a4paper,twoside,openright]{book}
%%\documentclass[12pt,a4paper,twoside,openright]{memoir}

\usepackage[msc,blankmsg]{costhesis}
%\usepackage[T1]{fontenc}
%%\usepackage{pslatex}
\renewcommand{\rmdefault}{ptm} 
\usepackage{mathptmx}
\usepackage[scaled=.90]{helvet}
\usepackage{courier}
%
\usepackage{bookmark}
\usepackage{caption}
\usepackage{subcaption}

%%----------------------------------------------------------------------------
%%   pcap2tex stuff
%%----------------------------------------------------------------------------
 \usepackage[dvipsnames*,svgnames]{xcolor} %% For extended colors
 \usepackage{tikz}
 \usetikzlibrary{arrows,decorations.pathmorphing,backgrounds,fit,positioning,calc,shapes}
 \usepackage{pgfmath}	% --math engine
%%----------------------------------------------------------------------------
%\usepackage[latin1]{inputenc}
\usepackage[utf8]{inputenc} % inputenc allows the user to input accented characters directly from the keyboard
\usepackage[swedish,english]{babel}
\usepackage{rotating}		 %% For text rotating
\usepackage{array}			 %% For table wrapping
\usepackage{graphicx}	 %% Support for images
\usepackage{float}			 %% Suppor for more flexible floating box positioning
\usepackage{color}           %% Support for colour 
\usepackage{mdwlist}
\usepackage{setspace}    %% For fine-grained control over line spacing
\usepackage{listings}		%% For source code listing
\usepackage{bytefield}    %% For packet drawings
\usepackage{tabularx}		%% For simple table stretching
\usepackage{multirow}	%% Support for multirow colums in tables
\usepackage{dcolumn}	%% Support for decimal point alignment in tables
\usepackage{url}	%% Support for breaking URLs
\usepackage{epstopdf}
\usepackage[perpage,para,symbol]{footmisc} %% use symbols to
                                %% ``number'' footnotes and reset
                                %% which symbol is used first on each
                                %% page

%%\usepackage{pygmentize}  %% required to use minted -- see python-pygments - Pygments is a Syntax Highlighting Package written in Python
%%\usepackage{minted}		%% For source code highlighting

\usepackage{hyperref}		
\usepackage[all]{hypcap}	 %% Prevents an issue related to
                                %% hyperref and caption linking
\usepackage[backend=biber, sorting=none]{biblatex}
\addbibresource{biblio.bib}

%% setup hyperref to use the darkblue color on links
\hypersetup{colorlinks,breaklinks,
            linkcolor=darkblue,urlcolor=darkblue,
            anchorcolor=darkblue,citecolor=darkblue}


%% Some definitions of used colors
\definecolor{darkblue}{rgb}{0.0,0.0,0.3} %% define a color called darkblue
\definecolor{darkred}{rgb}{0.4,0.0,0.0}
\definecolor{red}{rgb}{0.7,0.0,0.0}
\definecolor{lightgrey}{rgb}{0.8,0.8,0.8} 
\definecolor{grey}{rgb}{0.6,0.6,0.6}
\definecolor{darkgrey}{rgb}{0.4,0.4,0.4}
%% Reduce hyphenation as much as possible
\hyphenpenalty=15000 
\tolerance=1000

%% useful redefinitions to use with tables
\newcommand{\rr}{\raggedright} %% raggedright command redefinition
\newcommand{\rl}{\raggedleft} %% raggedleft command redefinition
\newcommand{\tn}{\tabularnewline} %% tabularnewline command redefinition

%% definition of new command for bytefield package
\newcommand{\colorbitbox}[3]{%
	\rlap{\bitbox{#2}{\color{#1}\rule{\width}{\height}}}%
	\bitbox{#2}{#3}}

%% command to ease switching to red color text
\newcommand{\red}{\color{red}}
%%redefinition of paragraph command to insert a breakline after it
\makeatletter
\renewcommand\paragraph{\@startsection{paragraph}{4}{\z@}%
  {-3.25ex\@plus -1ex \@minus -.2ex}%
  {1.5ex \@plus .2ex}%
  {\normalfont\normalsize\bfseries}}
\makeatother

%%redefinition of subparagraph command to insert a breakline after it
\makeatletter
\renewcommand\subparagraph{\@startsection{subparagraph}{5}{\z@}%
  {-3.25ex\@plus -1ex \@minus -.2ex}%
  {1.5ex \@plus .2ex}%
  {\normalfont\normalsize\bfseries}}
\makeatother

\setcounter{tocdepth}{3}	%% 3 depth levels in TOC
\setcounter{secnumdepth}{5} %% 3 sectioning levels. WARNING: command \mainmatter resets this field to its default value!!!
%%%%%%%%%%%%%%%%%%%%%%%%%%%%%%%%%%%%%%%%%%%%%%%%%%%%%%%%%%%%%%%%%%%%
%% End of preamble
%%%%%%%%%%%%%%%%%%%%%%%%%%%%%%%%%%%%%%%%%%%%%%%%%%%%%%%%%%%%%%%%%%%%

\iauthor{Antonios Kouzoupis}
\ititle{High performance shared state schedulers}
\isubtitle{}
\idate{2016}{July}{1}
\examinername{Associate professor Jim Dowling}
\supervisorname{Professor Seif Haridi}
\kthlogo{resources/images/KTH_logo.eps}
\itrita{YYYY}{NN}

\setlength{\headheight}{15pt}
\begin{document}

\frontmatter
\selectlanguage{english}
\begin{abstract}
\label{sec:abstract}
\setcounter{page}{1}

This is going to be my abstract

\end{abstract}
%%\clearpage
\selectlanguage{swedish}
%%\chapter*{Sammanfattning}
\begin{abstract}
\label{sec:swedish_abstract}

 Nu för tiden lagrar stora organisationer och forskningsinstitutioner enorma mängder data.
För att kunna utvinna någon värdefull information från dessa data behöver den bearbetas
av ett kluster av datorer. När flera datorer gemensamt ska bearbeta data behöver de utgå
från ett så kallat ``distributed processing framework''. I dagsläget är Apache Hadoop det
mest använda ramverket för distribuerad lagring och behandling av data. Detta examensarbete
är har genomförts vid SICS Swedish ICT där vi byggt Hops, en ny distribution av
Apache Hadoop som drivs av ett distribuerat MySQL Cluster NDB som erbjuder en hög tillgänglighet.
Hops-YARN är Hops ramverk för resurshantering med distribuerade ResourceManagers som lastbalanserar
deras ResourceTrackerService. I detta examensarbete använder vi Hops-Yarn på ett sätt där ``back-end''
databasen flitigt används för att hantera ResourceManagerns metadata och inkommande RPC-anrop. Vår
konfiguration erbjuder en hög feltolerans och återställer sig mycket snabbt vid
felberäkningar. Vidare används NDB-klustrets Event API för att ResourceManager ska kunna
kommunicera med den distribuerade ResourceTrackers.

Detta projekt syftar till att optimera de mekanismer som används för ihållande metadata
i NDB både i termer av transaktions begå tid men också i termer av pre-
bearbeta dem medan samtidigt garantera enhetlighet i RM: s tillstånd. ResourceManagerns tillstånd
i RAM-minnet får under inga omständigheter
avvika från det tillstånd som finns lagrat i NDB:n. Med dessa mål i åtanke undersöktes flera
lösningar som förbättrar prestandan och därmed gör Hops-Yarn jämförbart med Apache YARN.
De lösningar som föreslås i denna uppsats förbättrar “pure commit time” när en transaktion
görs i ett MySQL Cluster samt förbehandlingen och parallelismen i vår Transaction Manager.
Resultaten tyder på att Hops prestanda ökade dramatiskt vilket ledde till ett effektivare
nyttjande av tillgängliga resurser i ett kluster bestående av ett tusental datorer. När
nyttjandet av tillgänliga resurser i ett kluster förbättras med några få procent kan
organisationer spara mycket pengar.



\end{abstract}

\selectlanguage{english}
\begin{acknowledgements}

I would like to thank my examiner Jim Dowling for his invaluable
insights and crucial guidelines throughout this project and my
supervisor at KTH Seif Haridi. Also, I would like to express my
sincere gratitude to my supervisor at SICS Gautier Berthou for his
endless support, guidance and patience. He and his expertise was
the source of motivation and really helpful regardless the problem.

\end{acknowledgements}

\selectlanguage{english}
\tableofcontents

\listoffigures

\listoftables

%% add a list of listing if and listings are used
\lstlistoflistings

% \begin{notations}
% \end{notations}

\renewcommand\abbreviationsname{List of Acronyms and Abbreviations}
\begin{abbreviations}
\label{list-of-acronyms-and-abbreviations}

\begin{basedescript}{\desclabelstyle{\pushlabel}\desclabelwidth{10em}}
\item[YARN] Yet Another Resource Negotiator
  \cite{Vavilapalli:2013:AHY:2523616.2523633}
\item[RM] Resource Manager
\item[NM] Node Manager
\item[AM] Application Master
\item[RT] Resource Tracker
\item[HA] High availability
\end{basedescript}

\end{abbreviations}

\mainmatter
\setcounter{secnumdepth}{5} 
\chapter{Introduction}
\label{chap:introduction}
In the last years, the storage capacity of hard disk drives has
increased dramatically while at the same time their price has
decreased. Even though solid-state drives are still quite expensive,
big enterprises may benefit from the throughput they provide. This
trend of ``cheap'' storage solutions has led companies and research
institutes to store a volume of data that has never been stored
before. In 2014 Facebook was processing 600 TB daily
\footnote{https://code.facebook.com/posts/229861827208629/scaling-the-facebook-data-warehouse-to-300-pb/}
while according to rough estimates
\footnote{https://what-if.xkcd.com/63/} around 15 exabytes are stored
in Google's datacenters.

Another interesting area that is already generating a huge volume of
raw data is the DNA sequencing. According to \cite{10.1371/journal.pbio.1002195} the Sequence
Read Archive maintained by the United States National Institutes of Health
National Center for Biotechnology Information already contains more
than 3.6 petabytes of raw sequence data for a wide variety of samples
including microbial genomes, plant and animal genomes and human
genomes. As we can see in Figure \ref{fig:intro_genomics_growth} the
need for storage capacity will exceed the order of PetaBytes by the
year 2025.

\begin{figure}
\centering
\includegraphics[scale=0.5]{resources/images/Introduction/genomics_growth.png}
\caption{Growth of DNA sequencing \cite{10.1371/journal.pbio.1002195}}
\label{fig:intro_genomics_growth}
\end{figure}

It is clear by now that these volumes need a paradigm shift
from the traditional way we store and analyze data. It is not possible
anymore to store them in a single machine. We need to employ a
distributed system to harness the power of those data and extract
valuable results within a reasonal time frame.

%% Longer problem statement
%% General introduction to the area
\section{Problem description}
\label{sec:problem_description}
\section{Problem description}
\label{sec:problem_description}
This is the Problem description section.

\section{Problem statement}
\label{sec:problem_statement}
In Hops in order to improve performance and HA of the RM we have introduced an
in-memory distributed MySQL database which stores all the necessary
metadata. One great feature of HopsYarn is that the
\emph{ResourceTrackerService} (RT) of the RM is distributed into multiple
nodes in the cluster. That service is responsible for receiving and handling
heartbeats from the NMs. That way each instance handles only a portion of the total NM
heartbeats. The updated metadata are then stored into the database and
are streamed to the RM to update its view of the cluster. By load
balancing the ResourceTrackerService we have increased the performance
of the system while decreasing the load of the master RM which can
perform the rest of the operations without the load of handling every
single heartbeat.

Another equally important feature of HopsYarn is that RM stores
every event received and any scheduling decision into the MySQL
cluster. That makes our solution highly available with minimum
failover period. When a RM instance fails it re-builts the view of the
cluster by reading the latest state from the database. More details on
the architecture of both YARN and HopsYarn will be given in Chapter
\ref{chap:background} and \ref{chap:analysis}.

All these read/write operations to the database do not come without
a cost. First and foremost is the network latency. Even with high
throughput, low latency network in between of the RM/RT and the database, still
they take more time than in-memory operations. Especially in cases
where RM operations need more than one round-trip to the database, the
difference in performance is noticeable.

\section{Goals, Benefits, Ethics and Sustainability}
\label{sec:goals_ethics}
This is going to be my goals, benefits, ethics and sustainability section...

\section{Structure of this thesis}
\label{sec:thesis_structure}
This is the structure of this thesis

\chapter{Background}
\label{chap:background}
This chapter will give the reader the necessary background knowledge
in order for this work to be understandable. First, it will go through
Apache Hadoop, a distributed storage and processing framework. We will
give some brief introduction to Hadoop file system (HDFS), then we
will dive into the
resource manager (YARN) and in what way HopsYarn extend the Apache
YARN project. Later we will introduce a distributed, highly-available,
highly-redundant relational database, MySQL Cluster (NDB) and finally
we will give some insights on the different types of resource managing
systems.

In Figure \ref{fig:back_hpc_arch_overview} we can see a high level
overview of the architecture in HPC. Storage nodes are machines with very high disk capacity and bare
minimum processing power. Their main usage is to store data that are
going to be processed and analyzed in the future. The second building
block of the architecture is the Computing nodes. These machines have
no storage capabilities but they are equiped with the state of the art
processing units and a lot of RAM. Those modules communicate most
probably with a high throughput, low latency network. The most common
industry standard for interconnecting nodes in HPC is InfiniBand
\cite{infiniband} that it can
reach 30 Gb/s in each direction and sub-microsecond latency. Users
issue their jobs to the Head node which is responsible for transfering
the requested datasets from the Storage nodes to the Computing nodes,
monitor the tasks and finally return the result to the end user.

\begin{figure}
\centering
\includegraphics[scale=0.3]{resources/images/Background/hpc_arch_overview.png}
\label{fig:back_hpc_arch_overview}
\caption{HPC high lever architecture}
\end{figure}

In 2003 Google published a paper describing GoogleFS (GFS)
\cite{Ghemawat:2003:GFS:1165389.945450}, a proprietary distributed
file system. It was designed to run on large clusters of commodity
hardware, that are doomed to fail at some time. That was the main
motivation that drove GFS to be fault-tolerant and
highly-available. Apache HDFS is the open-source implementation of GFS
and it will be analyzed in Section \ref{ssec:hdfs}. In 2004 Google
published MapReduce \cite{Dean:2004:MSD:1251254.1251264}, a breakthrough programming model which exploited the locality
awareness of GFS and changed the way we process very big
datasets. MapReduce was later implemented for Hadoop and paved the way
for YARN, the current resource manager and scheduler 
which will be analyzed in Section \ref{ssec:yarn}.


\section{Hadoop}
\label{sec:hadoop}
This section will give a general intro to Hadoop platform.

\section{HDFS}
\label{sec:hdfs}
As this project is focused on the resource management framework, I will
not give a detailed description of the Hadoop Distributed File
System. Yet, I will go through some basic concepts that will make the
reader understand better the overall architecture of Hadoop.

HDFS is the distributed file system of Hadoop platform. It is designed
with the assumption that hardware failure is the norm and not an
exception making it highly fault-tolerant. Also, it is designed to run
on commodity, heterogenous, low-cost hardware making the setup and provisioning of
a cluster cheaper than HPC. HDFS has two main entities, the NameNode
(NN) and the DataNode (DN) and the architecture is depicted in Figure
\ref{fig:hadoop_hdfs}.

\begin{figure}
\centering
\includegraphics[scale=0.7]{resources/images/Background/hdfs_arch.png}
\label{fig:hadoop_hdfs}
\caption{HDFS architecture}
\end{figure}

\subsubsection{NameNode}
\label{sssec:nn}

Here I will briefly describe the NameNode

\subsubsection{DataNode}
\label{sssec:dn}

Here I will briefly describe the DataNode


\section{Data computation and Resource management}
\label{sec:resource_mgm}
So far we have discussed how to store datasets in the order of
terabytes and even petabytes in a distributed
and reliable way. We have gone through the most important ideas that
lay behind HDFS, the distributed file system of Hadoop, and how it
manages to overcome the fact that machines will fail and avoid
transferring of huge files into computation nodes in contrast to the HPC
architecture.

That is half the way of extracting valuable results out of big data
though. A cluster consists of thousands of physical machines with
certain resources in terms of CPUs, amount of RAM, network bandwidth,
disk space etc. We need a mechanism to harness all that power while at
same time exploiting the locality awareness to minimize data transfer
and maximize cluster utilization.

\subsection{MapReduce}
\label{ssec:mapreduce}
After the world wide web explosion at the ending of 1990's, Google has
emerged as one of the most significant web searching companies.
The novelty of Google was PageRank \cite{ilprints361}, an
algorithm counting the number of outgoing links of a webpage to determine its
importance. In order to apply the PageRank algorithm and form the
Google search results, first the webpage has to be scraped and
indexed. As of 2004 the raw size of the documents that had been
collected was more than 20 terabytes
\cite{Dean:2004:MSD:1251254.1251264}. Although the engineers at Google
have distributed and parallelized the algorithm, there were more tasks
that other teams have parallelized in a different way making it
difficult to maintain such a diverse codebase. That led them in 2004
to publish a paper about MapReduce, a generic framework to write distributed
applications that hide all the complexity of fault-tolerance, locality
awareness, load balancing etc.

MapReduce programming model borrows two very common functions from
functional programming, \emph{Map} and \emph{Reduce}. The \emph{Map}
function takes as input key/value pairs and produces as output a set
of key/value pairs as well. The \emph{Map} function is written by the
user and varies depending on the use case.

The \emph{Reduce} function, takes as input the
intermediate key/value pairs produced by \emph{Map} and merge them
together producing a smaller set of values. The \emph{Reduce} function
and the way it will merge the intermediate pairs is also provided by
the user.

A trivial example of the MapReduce programming model is that of
counting the occurrences of words in a text. The \emph{Map} function
takes as input a list of all the words in the text and emits tuples
in the form \texttt{(word,1)}, where \texttt{word} is every word
parsed. The result of the \emph{Map} function is passed to the
\emph{Reduce} function which adds the value of the tuples with the
same key, in that case is the word. The final result will be a list of
tuples with unique keys, where the key is all the words parsed from the text and the
value would be the occurrences of the word in the text.

Google provided a framework which took advantage of the locality
awareness of the already existing GFS and the MapReduce programming
paradigm. The execution overview of MapReduce is depicted in Figure
\ref{fig:mapreduce_execution_overview}. We can identify two entities
in MapReduce architecture, the \emph{Master} and the \emph{Workers}.

\begin{figure}
\centering
\includegraphics[scale=0.8]{resources/images/Background/mapreduce_exec_overview.png}
\label{fig:mapreduce_execution_overview}
\caption{MapReduce execution overview \cite{Dean:2004:MSD:1251254.1251264}}
\end{figure}

The \emph{Master} has the role of the coordinator that pushes the jobs
to the worker machines. It keeps track of the status of jobs in the
workers, informs other workers for the intermediate files produced
during the Map phase and pings the workers to verify their liveness.

The \emph{Workers} reside at the same physical hardware as the GFS
nodes to take advantage of the data locality. They are divided into
\emph{mappers}, which execute the Map function and \emph{reducers},
which perform the reduce phase as instructed. Workers are
pre-configured with available map or reduce slots depending on their
CPU or RAM.

At the very beginning, a user submits a job to Master. Master forks
the submitted job and is responsible to schedule the forks on workers
that $(a)$ have available map/reduce slots and $(b)$ have the
requested datasets stored locally. Upon the scheduling is done, the
Map phase begins in the mappers. They read the datasets from the local
hard drive and perform the Map function. Master periodically pings the
mappers to get informed about the status of the job and the health of
the node itself. When a mapper node completes its task, it writes the
intermediate key/value pairs to the local file system and informs the
Master node. The Master node in turn, notifies the reducer nodes that
an intermediate result is available at a specific node, where the
latter reads it (the result) remotely and perform the reduce
function. Finally, when all the reducers have completed the Reduce
phase, the Master notifies the user program.

\subsubsection{MapReduce Fault Tolerance}
Primary concern of the engineers was the fact
that machines will eventually fail. MapReduce will run on a cluster of
thousands of machines so the probability of a failed one would be
higher. For that reason they equipped MapReduce with a heartbeating
mechanism in order to be able to handle such situations.
The Master periodically pings the workers. The workers should respond back
within a predefined timeout before they are declared dead. When a node
that performs the Map phase is declared dead, the job that was
running at that node is set to \emph{idle} and is rescheduled on
another node. Similarly, when a map job has finished, since the
intermediate result is written to the local hard drive, the job has to
be rescheduled in a different machine. When a reducer node has failed
and the job is still in \emph{running} state, then it is set back in
\emph{idle} state and assigned to another node. In case of a completed
Reduce phase, the result is stored in the global file system, GFS in
that case. So, even with a failed reducer machine, the result will
still be available and the job should not be rescheduled.

While a Worker failure does not greatly affect the MapReduce job, it
is not the same case with a Master failure. If a machine that is a
Master node fails, then the whole MapReduce job is canceled and the
client is informed so that it can retry later on. ``However, given
that there is only a single master, its failure is unlikely;''
\cite{Dean:2004:MSD:1251254.1251264}.

\subsubsection{Limitations}
\label{sssec:mapreduce_limitations}
MapReduce facilitated engineers to ``easily'' write parallel data
processing applications by hiding all the complexity of a distributed
system. It provided some sort of fault tolerance and it was generic
enough to fit in various domains.

MapReduce and Hadoop over the years has become the industry standard
for processing and storing big volumes of data. After some period of
heavy usage it became clear that, although the platform itself suited
the needs for distributed, reliable storage and cluster management,
there were some limitations that had to be addressed. The two key
shortcomings were regarding the tight coupling of a programming model
with the resource management infrastructure and the centralized
handling of jobs \cite{Vavilapalli:2013:AHY:2523616.2523633, 6680946}.

A user who wantsto write an application for MapReduce framework,
all it has to do is to provide implementation for the two
first-order functions \emph{Map} and \emph{Reduce}. This static
map-reduce pipeline is very limiting though, as every job should have exactly one
Map function followed by an optional Reduce function. That workflow is
not suitable for large scale computations, such as machine learning
programs that require multiple iterations over a dataset. That means
that multiple individual MapReduce jobs have to be scheduled while the
frequent write of data in disk or in a distributed file system would
impose a considerable latency. A common pattern/misuse
\cite{Vavilapalli:2013:AHY:2523616.2523633} was to submit jobs with a
map phase only that spawned alternative frameworks or even web
servers. The scheduler had no semantics about the job except that they
were map jobs with a consequence in the cluster utilization, creating
deadlocks and a general instability to the system. The second drawback
of MapReduce and Hadoop 1.x was the centralized job handling and
monitoring of their flow. The Master or the \emph{JobTracker} should
monitor every single job, receiving liveness heartbeats, resource
requests etc. The is a heavy workload for a single machine that drove
to major scalability issues.

These two crucial limitations of MapReduce led to a total re-design of
Hadoop. Since Hadoop 2.0 there is a resource management module, YARN --
\emph{Yet Another Resource Negotiator} which will be analyzed in
section \ref{ssec:yarn} and MapReduce is just another application
running on a cluster of physical machines.

\subsection{YARN}
\label{ssec:yarn}
Considering the limitations outlined in section
\ref{sssec:mapreduce_limitations}, Vinod Kumar Vavilapalli et
al. presented YARN \cite{Vavilapalli:2013:AHY:2523616.2523633} the new
resource management layer that was adopted in Hadoop 2.0. The new
Hadoop stack now is depicted in Figure \ref{fig:yarn_hadoop1_hadoop2_arch} where YARN is the
cluster resource management module and MapReduce is one out of plenty
applications running on top of YARN. This architectural transformation
paved the way for a wide variety of frameworks like
Apache Spark \cite{apache_spark}, Apache Flink \cite{apache_flink},
Apache Pig \cite{apache_pig}, etc to run on the
Hadoop platform like any other YARN application.

\begin{figure}
\centering
\includegraphics[scale=0.6]{resources/images/Background/hadoop1_hadoop2_arch.png}
\label{fig:yarn_hadoop1_hadoop2_arch}
\caption{Hadoop 2.0 stack \cite{hortonworks_hadoop_stack}}
\end{figure}

The new architecture of Hadoop 2.x separates the resource management
functions from the programming model. It delegates the
intra-application communication and the tracking of the execution flow
to per-job components. That unlocks great performance improvements,
improves scalability and enables a wide variery of frameworks to share
the cluster resources in a very gentle way.

YARN uses three main components to provide a scalable and fault
tolerant resource management platform. The first component is the
\emph{ResourceManager} (RM), a per-cluster daemon that tracks resource
usage and node liveness and schedules jobs on the cluster. The second
component is a per-node \emph{NodeManager} (NM) which is responsible
for monitoring resource availability on the specific node, reporting
faults to RM and managing container lifecycle. Finally, there is the
\emph{ApplicationMaster} (AM) which coordinates the logical plan of a
single job, manages the physical resources offered by the RM and
tracks the execution of the job. A high level overview of YARN
architecture is described in Figure \ref{fig:yarn_arch_overview}. RM
has a global view of the cluster and provides the scheduling
functionality, while the per-job AM manages the dynamic resource
requests and the workflow of the tasks. Containers that are allocated
by the RM are locally managed by the NM in each node in the cluster.

\begin{figure}
\centering
\includegraphics[scale=0.5]{resources/images/Background/yarn_arch_overview.png}
\label{fig:yarn_arch_overview}
\caption{YARN architecture overview \cite{Murthy:2014:AHY:2636998}}
\end{figure}

\subsubsection{ResourceManager}
\label{sssec:rm}
In YARN the RM acts as the central authority for allocating resources
in the cluster. It works closely with the per-node NodeManager getting
an updated view of the cluster by the heartbeats received. The RM
itself allocates generic resources in the cluster in the form of
\emph{containers} that have specific CPU and RAM requirements. Those
resource requests are piggybacked in the heartbeats issued by every AM.
As RM is completely unaware about the job execution plan,
it is up to the AM to make local optimizations and assign the
resources accordingly. RM internally consists of several modules but
the three most important are the \emph{ApplicationMasterService}, the
\emph{ResourceTrackerService} and the \emph{Yarn Scheduler} as shown
in Figure \ref{fig:yarn_RM_components}.

\begin{figure}
\centering
\includegraphics[scale=0.5]{resources/images/Background/RM_components.png}
\label{fig:yarn_RM_components}
\caption{ResourceManager components}
\end{figure}

The \emph{ApplicationMasterService} is responsible for receiving and
handling heartbeats from the AMs that are launched in the
cluster. Heartbeats are designed to be as compact as possible, still
not excluding any vital information. For that reason, Google Protocol
Buffers \cite{proto_buf} are used for every communication among YARN
components. Protocol Buffers is a language-neutral, platform-neutral
mechanism for efficiently serializing data. The heartbeat mechanism
serves both as a scalable way for the RM and AM to communicate, but
also for the RM to track the liveness of AMs. \emph{ResourceRequests}
contain information such as the resources per container in terms of
virual cores and memory, the number of containers, locality
preferences and priority of requests. The scheduler then tracks,
updates and satisfies these requests with available resources in the
cluster. The RM builds the view of the cluster with the available
resources from the information it receives from the NMs. The scheduler
tries to match the locality constraints as much as possible and
responds back to AM with the allocated containers along with
credentials that grant access to them. The RM also keeps track of the
AM health through the heartbeats received. The component that handle
the liveness property of every AM is the \emph{AMLivenessMonitor}. In
case of a missed heartbeat, then that particular AM is deemed dead and
is expired by the RM. All the containers that were allocated for
that AM are marked as dead and the RM reschedules the same application
(ApplicationMaster) on a new container.

\subsubsection{ApplicationMaster}
\label{sssec:am}
ApplicationMaster stuff...

\subsubsection{NodeManager}
\label{sssec:nm}
NodeManager stuff...

\subsubsection{YARN HA \& fault tolerance}
\label{sssec:yarn_ha}
YARN HA and fault tolerance stuff...


\section{MySQL Cluster}
\label{sec:ndb}
Give some background info regarding MySQL cluster NDB

\section{Hops-YARN}
\label{sec:hopsyarn}
Hops-YARN is a drop-in replacement of Apache Hadoop YARN for the Hops
\cite{hops} platform. From a user's perspective there is no difference
between the two implementations and an Apache YARN application can be
scheduled on Hops-YARN without any modification. Although the
interface is the same, there are some key characteristics that
distinguish the two implementations and can be categorized into
architectural, recovery mechanism and load balancing. In the rest of
the section I will present the differences in every category.

\subsection{Architecture}
\label{ssec:hopsyarn_arch}
In Hops we heavily use a MySQL Cluster that was briefly introduced in
Section \ref{sec:ndb}. We store all kinds of metadata spanning from
Hops-YARN to Hops-HDFS, a new distribution of Apache HDFS, and
HopsWorks, a web-based UI front-end to Hops. The fact that everything
is stored in the database leverages the limited amount of information
that can be stored in the JVM heap of a single machine and opens up
great opportunities of improvement and experimentation.

Apache Hadoop uses ZooKeeper to detect failures and elect a new leader.
Since the MySQL Cluster is already in place storing data, we use a
leader election mechanism proposed by Salman Niazi et
al. \cite{Niazi2015} that uses NewSQL databases in a novel way.
The protocol guarantees a single process acting
as a leader at any point of time with performance comparable to
Apache ZooKeeper. Having the database acting as a persistent storage
and as a leader election mechanism, Hops drops ZooKeeper from its
stack releaving the operations team from the burden of maintaining
one extra service.

\subsection{Fault tolerance \& HA}
\label{ssec:hopsyarn_fault_tol_ha}
In Section \ref{sssec:yarn_ha} I have outlined how Apache Hadoop YARN
deals with RM failures and provides a highly available solution. In
Hops-YARN we follow a different path for storing information for
recovery. In YARN, AMs and NMs communicate with the RM through a
heartbeating mechanism. These heartbeats carry information such as
(de)allocation requests, health status, etc Since the database allows
for millions of transaction per second, we store every single RPC that
the RM receives and delete them when the request is handled. Moreover,
every operation that is done on the scheduler state is reflected on a
modification in the database. The main advantage of this approach over
the approach followed by Apache YARN is in terms of recovery
time. It is much faster to read the complete state of the scheduler from the
distributed in-memory database than asking from every NM to re-sync
and send back a list of all running containers. Particularly when
the cluster size grows in the order of thousands of
machines. Moreover, in case of a crash in Apache YARN, the RM
instructs all the AMs and NMs to re-sync and send again any request
that has been sent but not handled. In Hops-YARN, the RM recovers the
unhandled RPCs from the database and replays them.

In terms of HA the architecture of Hops-YARN is basically the same
with an Active/Standby model for the scheduler, although some
improvements have been made for the Standby nodes described in the
following section.

\subsection{Load balancing}
\label{ssec:hops_yarn_load_balance}
Standby is boring! Except for being boring, having a physical machine
idle for most of the time is a waste of resources. Although RM is a
monolith, its architecture is modular. The components of the RM are illustrated in
Figure \ref{fig:yarn_RM_components}. Hops-YARN follows a very original
approach of distributing the \emph{ResourceTrackerService} among the
StandBy RM nodes. The \emph{ResourceTrackerService} is responsible for
handling the RPCs from the NMs (see Section \ref{sssec:rm}). Assume a
cluster with the moderate size of 5000 nodes and the default value of 1
second for the heartbeat interval. That implies that every second the
RM should handle 5000 RPCs just for keeping track the NM status. In
Hops-YARN, the StandBy RMs also run the
ResourceTrackerService. When NMs register with the RM they are
assigned to the least overloaded \emph{ResourceTracker} (RT) -- StandBy
\emph{ResourceManager}. The information received by each
\emph{ResourceTracker} separately is stored in the
database and through the event API of NDB is streamed to the Active
RM to update its view of the cluster. In that case, NDB serves as a
communication channel between the RT and the RM. With that
architecture the load of tracking 5000 nodes is distributed among all
the RMs in the cluster. An overview of Hops-YARN distributed
ResourceManager is illustrated in Figure \ref{fig:hopsyarn_dist_rm}.

\begin{figure}
\centering
\includegraphics[scale=0.5]{resources/images/Background/hopsyarn_arch_overview.png}
\label{fig:hopsyarn_dist_rm}
\caption{Hops-YARN distributed RM architecture}
\end{figure}

\section{Taxonomy of schedulers}
\label{sec:taxonomy_of_schedulers}
Taxonomy of schedulers monolithic, two-level, shared state

\chapter{Methods}
\label{chap:methods}
The aim of this project is to minimize the time spent for Hops-YARN to commit
a transaction in the MySQL Cluster while at the same time parallelize the
process. This will increase the number of events processed by the
scheduler and the overall cluster utilization. In order to explore
the limits of the current system, different optimization techniques
were examined and evaluated as presented in Chapters
\ref{chap:implementation} and \ref{chap:analysis}. For that reason a
\emph{Quantitative Research} method was followed. The impact of a new
feature in the system was measured in terms of commit time in the
database, the percentage of events handled by the ResourceManager or
the overall cluster utilization as suited.

The quantitative research method is supported by \emph{deductive}
approach and \emph{experimental} method. Before starting my endeavor into
implementing a new feature, I made a thorough profiling of the
workflow of the system in order to identify the bottlenecks. Afterwards,
wherever it was possible, I created a prototype of the new
feature and did a micro-benchmark to validate any performance
gain. The micro-benchmark gave us an incentive on whether it was worth
investing on that feature or not. In case of a positive feedback, I
implemented a concrete version of the feature and finally measured the system's
performance. After the completion of one feature, I iterated the
procedure explained above to identify more bottlenecks in the system.

Regarding the data collection method, \emph{experiments} were
conducted throughout the project duration and with \emph{statistics}
data analysis I calculated the results presented in Chapter
\ref{chap:analysis}. For the purpose of data collection I used a
cluster operated by SICS with seven physical machines and two MySQL
Clusters with four and two nodes respectively. In order for the
results to be \emph{reliable} I run each experiment several times and
made an average where it was reasonable. Since the experiments took a
some time to finish and produced a lot of valuable data, I used custom
code to dump these data to files and process them later. For the
\emph{validity} part of the experiments, I used a simulator that
simulated a configurable number of nodes in a Hadoop cluster and
measured in fixed intervals the variables that were interesting for
the system performance. After spawning the ResourceManager and the
ResourceTracker in multiple machines, it starts simulating the
NodeManagers that heartbeat the scheduler. Also, it parses existing
trace files and issues a synthetic workload to the scheduler. The simulator I used is a
modified version of the simulator that ships together with the Hadoop
distribution. The difference is that the load of simulating the
NodeManagers and sending application launching requests is distributed across
multiple machines in the cluster. Since the workload is parsed from
trace files, it makes my experiments \emph{reproducible} by other
researchers. A note should be taken here. The performance of the system is affected
by the load of the physical machine, the network traffic and the
utilization of the database, so a small variation on the results
should be anticipated.


\chapter{Implementation}
\label{chap:implementation}
Implementation details intro will go here


\section{Foreign key constraints}
\label{sec:fk_constraints}
In Hops for reasons that I have outlines in previous chapters we
persist all the metadata in our persistent storage solution. MySQL
Cluster is a relational distributed database, so data is stored into
tables with very specific properties. Since version 7.3.1, MySQL
Cluster supports foreign key constraints. Foreign keys is a powerful
feature of relational databases that guarantee some kind of
consistency between two tables. One important aspect of foreign keys
is that they map relationships of the ``real-world'' into
relationships in the database.

The database schema of Hops consists of 95 tables, 64 out of them are
used by Hops-YARN. The information they store span from incoming RPCs
to scheduler state and nodes' statuses. We make heavy use of foreign
key constraints, mainly the \texttt{ON DELETE} referential action, to
ensure integrity when a row from a parent table is deleted. Although
the database schema of Hops-YARN is too big to fit in
a single page, Figure \ref{fig:impl_fk_yarn_schema}, the number of
foreign key relationships -- they are illustrated with a solid line --
is clear.

*** Say why this is bad in terms of performance. Subsection for each
parent table and how I dealt with it

\begin{figure}
\centering
\includegraphics[scale=0.2, angle=90]{resources/images/Implementation/hops_yarn_ndb_schema_full.png}
\label{fig:impl_fk_yarn_schema}
\caption{Hops-YARN database schema}
\end{figure}

\section{Transaction state aggregation}
\label{sec:tx_aggregation}
In this section I will present the new architecture of the Transaction
State manager of Hops-YARN. A \emph{Transaction State} in Hops-YARN is
an object that holds all the information that should be persisted in
the database back-end in a consistent way. Updates in the RM state are
generated either from heartbeats received from NMs and AMs or from
events that are streamed from the NDB event API (see Section
\ref{ssec:hops_yarn_load_balance}). Every modification in the
scheduler state should be reflected with updates in the corresponding
tables in NDB. Such modifications include:
\begin{itemize}
\item New applications that have been issued to the scheduler through
the YARN client. This include name of the application, user of
the application, the Application Submission Context, state of the
application etc.

\item New application attempts including reference to AM,
diagnostics, list of nodes that the containers of this application run
etc.

\item Newly created containers that applications will use

\item Containers that have finished their job and should be removed
from scheduler's state

\item Containers that have updated their state for some reason etc.

\item Newly added nodes in the cluster
\end{itemize}

For the full list of the modifications tracked and persisted in NDB,
consult \texttt{TransactionStateImpl.java}
\footnote{\url{https://goo.gl/Ukq4Tp}} file.

It is important that the state persisted in the database reflects the
state of the ResourceManager in memory. In case of a recovery, the
state recovered should be the last known state and be consistent. An
example of inconsistency would be a container to be listed as running
even though the application that was using it has finished. In order to
avoid such situations and achieve the consistency level we desire
there is the notion of Transaction State (TS). A TS is committed in an
``atomic'' fashion. This is facilitated by the transactional nature of
MySQL Cluster, so a commit of one TS is one ``big'' SQL
transaction. Using transactions we achieve isolation and atomicity of
our Hops TS. With the all-or-nothing architecture of transactions, if
the RM crashes in the middle of a commit, then the whole TS will be
aborted. Similarly, in case of an error during the commit phase it will
roll-back the whole transaction leaving the database in a clean state.
More information regarding how a TS is created is presented in Section
\ref{ssec:impl_batch_system}. Having our safety property covered, one
more reason to use Transaction States is for efficiency. Committing a
single Transaction State with a lot of modifications is more efficient
than committing small modifications several times.

The Transaction State is implemented generally with concurrent hash maps and
accessors in the style of \texttt{add*ToAdd} for entries that we
want to persist in the database and \texttt{add*ToRemove} for entries
that we want to delete from the database. Even inside a TS we should
be very careful on how we access the fields. A TS holds modifications
from several RPC requests that may modify the same element. For
example one RPC might create a new container and the next one to
destroy it. YARN internals are event-based. So there is no guarantee
about the order that events will be processed. The second RPC might
follow a different code path and finish earlier than the first one. In
that case what will be persisted in the database would be the creation
of a new container, something completely wrong. That is the reason
we hold two separate data structures for the same table, one for
insert operations and another for delete operations. First the insert
data structure is processed that persists entries in the database and
then the delete data structure.

In Section \ref{ssec:impl_batch_system} I will give some insights on
the existing batching system while on Section
\ref{ssec:impl_aggr_mechanism} I will explain the new queuing
mechanism that improved the commit time.

\subsection{Batching system}
\label{ssec:impl_batch_system}
So far we have discussed why we need the Transaction State in
Hops-YARN and how it is implemented. We still miss the part of how
Hops-YARN handles heartbeats or events from NDB and how the scheduler
updates the fields in it. A TS object is managed by the
\emph{TransactionStateManager} a custom service that runs on Hops-YARN.
It is responsible for creating new TS, provide the
current TS to methods that have requested it and keep track of how many
RPCs have requested the TS.

Heartbeats from AMs and NMs from the side of RM are perceived as
RPCs. Upon an RPC, the method invoked will ask for the TransactionStateManager
to provide the current TS. The TS will be piggybacked to every event
triggered by RM components and ``travel'' all the way until that
request has been fully handled. Each modification made by the components
to the state of the RM will also be added to the insert and delete
data structures explained above. TransactionStateManager service
provides isolation by batching several RPC requests together, let the
requests be handled and then batch the next RPCs in a different
TS. The activity diagram of the batching system is depicted in Figure
\ref{fig:impl_rpc_batch_system}.

\begin{figure}
\centering
\includegraphics[scale=0.4]{resources/images/Implementation/rpc_batch_system_activity.png}
\caption{Activity diagram of RPC batching system}
\label{fig:impl_rpc_batch_system}
\end{figure}

At the beginning the TransactionStateManager service creates a
Transaction State object. Heartbeats start coming from the
ApplicationMasters and the NodeManagers in the form of RPCs. The
methods that are invoked, get the current TS from the
TransactionStateManager and piggyback them to the events
triggered, while various RM components handle those events in separate
threads.
The TransactionStateManager service keeps receiving RPCs
until a certain threshold of received RPCs -- default is 60, or until
a timeout has been reached -- default is 60 ms. At the point where
either of two is satisfied, it blocks responding to further
\texttt{getCurrentTransactionState} requests until all the RPCs in the
previous batch have been handled properly and put in the queue to be
committed in the database back-end. When this is done, it
creates a new Transaction State object and unblocks the receiving of
new RPCs. At the same time but in a \textbf{different thread}, the previous
Transaction State is committed to the database as described in Section
\ref{ssec:impl_aggr_mechanism}.

In Hops-YARN we have distributed the \emph{ResourceTrackingService} to
multiple machines in the cluster. They receive heartbeats from the
NodeManagers and persist them in NDB. MySQL Cluster has an event API
that can stream events to subscribers. In RM there is a custom C++
library that receives the events emitted from NDB, creates a Java
representation of them and put them in FIFO queue. In the background
there is a Hops service that polls from the queue and triggers the
appropriate events. The Transaction State is also included in these
events as the cascading modifications should be persisted in the
database. The same Transaction State manager service is used as
described in the previous paragraph, so from the manager's perspective
there is no difference between RPC events and NDB events.

\subsection{Aggregation mechanism}
\label{ssec:impl_aggr_mechanism}
Heartbeats arrive in the RM and invoke specific methods. The methods
invoked create specific events which include the TS and are sent to an
event dispatcher thread. The event dispatcher will forward the events
to the appropriate RM components that will trigger some actions and
probably create more events. All along the ``journey'' of these
events, TS track all the necessary modifications that should be
persisted in the database back-end. After all events in a batch have
been properly handled they should be committed in NDB.

Persisting in a non-volatile storage solution is expensive due to
exclusive locks, OS buffers, I/O interrupts etc. On top of that, if you are
persisting data in a remote database, then the network introduces a
few milliseconds of latency as well. For that reason, when a TS is ready to be committed, it forks
a new thread which will handle the actual database operations to
persist its state. Having the commit mechanism parallelized, multiple
TS can be persisted concurrently. At that point, we have just
invalidated our consistency model. Sooner or later we will
reach the case where two TS would be committed in the wrong order
corrupting the state of the database. The mechanism explained in
Section \ref{sssec:impl_aggr_old} controls the commit phase of each TS
guarantying that two conflicting TS will be committed in the correct
order. In Section \ref{sssec:impl_aggr_new} I will outline the
shortcomings of the existing mechanism and describe the new mechanism
I build that extends the previous.

\subsubsection{One TS per commit}
\label{sssec:impl_aggr_old}
The mechanism that controls the commit phase of TS should allow as
much as possible parallelism without violating our constraints. Our
constraints are:
\begin{itemize}
\item If the RPCs that are batched in Transaction State $TS_0$ have
  been fully handled before the RPCs batched in $TS_1$ and they both
  modify the same \textbf{YARN Application}, then the commit phase of $TS_0$
  should have been successfully completed \textbf{before} the commit
  phase of $TS_1$ begins

\item If the RPCs that are batched in Transaction State $TS_0$ have
  been fully handled before the RPCs batched in $TS_1$ and they both
  modify the same \textbf{NodeManager}, then the commit phase of $TS_0$
  should have been successfully completed \textbf{before} the commit
  phase of $TS_1$ begins

\item If Transaction States $TS_0$ and $TS_1$ modify different YARN
  Application \textbf{and} different NodeManager then they can be committed in parallel
\end{itemize}

Every TS keeps a list with the IDs of the applications it modifies and
a list with the IDs of the nodes it modifies. In the commit mechanism,
there is a FIFO queue for each and every application ID and node
ID. Before committing a TS, the mechanism puts it in the corresponding
queues both for the applications and the nodes it modifies. Then it
gets the IDs of the applications and nodes it modified. In order for
the TS to be committed, it should be in the \textbf{head} of the
queues for the corresponding application IDs \textbf{and} node IDs. If
it is not in the head of one application or node queue, it means that
a previous TS modified the same application or node and should be
committed before. The un-committed TS is put in a queue to be
examined later.

To make it more clear, I will give an example of how the commit mechanism
works. To make it easier, I assume that the only lock is on
application IDs and we have only three applications running. The FIFO
queues for the applications are depicted in Figure
\ref{fig:impl_tx_aggr_queue}. The initial state of the system is in
Figure \ref{fig:impl_tx_aggr_sub0} where no TS has been handled yet
and the queues are empty. Technically, the queues for the application
IDs and node IDs are created lazily when the TS has been fully handled
and is ready to be committed. RPCs for \texttt{TS\_0} have been
handled and is ready to be committed in the database. It modifies
entries for applications with ID \texttt{app\_0} and \texttt{app\_2},
so it is placed in the respective queues, Figure
\ref{fig:impl_tx_aggr_sub1}. \texttt{TS\_0} is the head
of the queues for the applications it modifies so it can start the
commit phase. Since a commit phase may fail and roll-back, it will be
removed from the queues when the transaction has been successfully
completed.

While \texttt{TS\_0} tries to persist its modifications to the
database, two more Transaction States have finished and are put in the
queue to be committed, Figure
\ref{fig:impl_tx_aggr_sub2}. \texttt{TS\_1} modifies applications
\texttt{app\_0} and \texttt{app\_1} while \texttt{TS\_2} modifies
applications \texttt{app\_1} and \texttt{app\_2}. The commit mechanism
checks whether \texttt{TS\_1} can be committed. Remember that
\texttt{TS\_0} is still committing to the database. The mechanism
fetches the applications that \texttt{TS\_1} modifies and checks if it
is the head in each queue. For \texttt{app\_1} it is in the head of
the queue but not for \texttt{app\_0}, \texttt{TS\_0} should finish
first. So it will be examined after \texttt{TS\_0} is done. The same
applies for \texttt{TS\_2}.

\texttt{TS\_0} is now complete and removed from the queues of the
applications it modified, Figure
\ref{fig:impl_tx_aggr_sub3}. \texttt{TS\_1} is now the head for
\texttt{app\_0} and \texttt{app\_1} so it starts the commit
phase. This is not the case though for \texttt{TS\_2} which still has
to wait for \texttt{TS\_1} to finish. \texttt{TS\_1} successfully
completes its transaction, removed from the queues and now
\texttt{TS\_2} is in the head position, Figure
\ref{fig:impl_tx_aggr_sub4} so it can start committing to the database.

\begin{figure}
  \centering
  \begin{subfigure}[t]{0.3\textwidth}
    \includegraphics[scale=0.4]{resources/images/Implementation/commit_system_0.png}
    \caption{}
    \label{fig:impl_tx_aggr_sub0}
  \end{subfigure}
  \hfill
  \begin{subfigure}[t]{0.3\textwidth}
    \includegraphics[scale=0.4]{resources/images/Implementation/commit_system_1.png}
    \caption{}
    \label{fig:impl_tx_aggr_sub1}
  \end{subfigure}
  \hfill
  \begin{subfigure}[t]{0.3\textwidth}
    \includegraphics[scale=0.4]{resources/images/Implementation/commit_system_2.png}
    \caption{}
    \label{fig:impl_tx_aggr_sub2}
  \end{subfigure}
  \\[2em]
  \begin{subfigure}[t]{0.3\textwidth}
    \includegraphics[scale=0.4]{resources/images/Implementation/commit_system_3.png}
    \caption{}
    \label{fig:impl_tx_aggr_sub3}
  \end{subfigure}
  \qquad
  \begin{subfigure}[t]{0.3\textwidth}
    \includegraphics[scale=0.4]{resources/images/Implementation/commit_system_4.png}
    \caption{}
    \label{fig:impl_tx_aggr_sub4}
  \end{subfigure}

  \caption{Example of TS for the queue to be committed}
  \label{fig:impl_tx_aggr_queue}
\end{figure}

\subsubsection{Multiple TS per commit}
\label{sssec:impl_aggr_new}
The existing mechanism described in the section above provided the
consistency model we required and parallelism for
non-conflicting TSs. The major issue that had to be addressed was that
for conflicting TSs $(1)$ they had to wait in the queue for another TS
to finish, in the example above \texttt{TS\_1} and \texttt{TS\_2} and
$(2)$ for every transaction committed we paid the network latency
penalty and the time probably to acquire locks etc. In order to solve
the first problem we examined different solutions. First I tried to
make a more fine-grained lock. Instead of locking on application and
node IDs, try to lock for containers. Although this solution would
increase parallelism, it was very risky that we would end-up with a
corrupted state. Next solution was to remove the system with the
queues and replace it with exclusive reentrant locks. The expected
result was to increase performance but at the end it was the same.

The solution proposed in this section and implemented reduces both the
wait time in the queue for a TS to be committed and the RTT for each
transaction performed. The mechanism extends the method described in
Section \ref{sssec:impl_aggr_old} by aggregating multiple TSs into a
single one while it guarantees consistency. The queue system still
gives us the proper order in which the TSs should be persisted. At the
beginning a TS
is examined if it should be persisted in the database. If it is not
possible to be persisted due to conflicting TSs then it is put in the
\emph{toBeAggregated} set. If it is permitted to commit, then it does so
and constructs an extended TS called
\emph{AggregatedTransactionState}.

The \emph{AggregatedTransactionState} contains TSs from the
\emph{toBeAggregated} set that are eligible for commit according to the
following aggregation rules:
\begin{enumerate}
  \item A TS was not the head in its respective queues at the time it
    was examined for commit, but until now the conflicting TS(s) have
    been committed and removed. So now it is in the head of the
    queues.

  \item A TS is still not in the head of the queues, but the
    conflicting TSs that should be committed before, have been already
    aggregated in the \emph{AggregatedTransactionState}.
\end{enumerate}

The first rule is trivial and we have examined it in the previous
section. For the second rule, the mechanism gets the modified
application and node IDs from a Transaction State \texttt{TS\_a}.
Then it retrieves all the conflicting TSs from the appropriate queues
that are blocking \texttt{TS\_a} from being committed. If all of the
conflicting TSs have already been aggregated -- put in the
\emph{AggregatedTransactionState}, then the mechanism aggregates
\texttt{TS\_a} as well and it proceeds by examining the next TS in the
\emph{toBeAggregated} set. At the end of this process we will end-up
with a ``big'' Transaction State, the
\emph{AggregatedTransactionState} that will be committed in the
database. The \emph{AggregatedTransactionState} class actually extends
the \emph{TransactionState} class and for every TS that is aggregated
it updates the data structures with the modifications of that TS. The
\emph{toBeAggregated} data structure is a FIFO queue so the TSs that
are examined for aggregation are kept in the correct order. At the
end of the aggregation process, the data structures of the
AggregatedTransactionState hold the correct, most recent modifications to be persisted.

Consider the state of the queues as in Figure
\ref{fig:impl_tx_aggr_sub2}. Both \texttt{TS\_1} and \texttt{TS\_2}
cannot be committed because they are blocked by \texttt{TS\_0} so they
are added to the \emph{toBeAggregated} queue. At some
point \texttt{TS\_0} is committed in the database, removed from the
queues as in Figure \ref{fig:impl_tx_aggr_sub3}, creates the
\emph{AggregatedTransactionState} (ATS)
and starts the aggregation process. \texttt{TS\_1} is the first
candidate for aggregation. At that point \texttt{TS\_1} is at the head
of its respective queues and following the aggregation rule $1$ it is
aggregated. All the data structures of \texttt{TS\_1} are copied to
the data structures of the ATS. Next in the \emph{toBeAggregated} set
is \texttt{TS\_2}. \texttt{TS\_2} is not the head in the queue for
application \texttt{app\_1} but the conflicting TS, \texttt{TS\_1} has
already been aggregated. So according to aggregation rule $2$,
\texttt{TS\_2} is also aggregated, probably updating the values that
\texttt{TS\_1} has put in TSA. The \emph{toBeAggregated} set is now
empty and the ``big'' AggregatedTransactionState is committed to the
database. Compared to the existing commit mechanism, where we have
done three commits in the database, we have now reduced it to two.

\begin{figure}
\centering
\includegraphics[scale=0.5]{resources/images/Implementation/commit_system_aggr_example.png}
\caption{Aggregate commit mechanism example}
\label{fig:impl_tx_aggr_example}
\end{figure}

Another case that demonstrates the improvement we have achieved is
described in the following example. Consider the locking queues as
depicted in Figure \ref{fig:impl_tx_aggr_example} for two applications. All the Transaction
States are blocked behind \texttt{TS\_0} which is at the commit
phase. At that time the \emph{toBeAggregated} set contains
\texttt{TS\_1}-\texttt{TS\_5}. \texttt{TS\_0} finishes its commit phase, removes
itself from the locking queues and begins the aggregation phase. First
it examines \texttt{TS\_1}. It is in the head of the queue for
application \texttt{app\_0} and according to aggregation rule $1$ it
should be aggregated. Next in the \emph{toBeAggregated} set is
\texttt{TS\_2}. The conflicting state is \texttt{TS\_1}, which is
already aggregated so according to aggregation rule $2$ it is
aggregated too. \texttt{TS\_3} is in the head of the queue for
\texttt{app\_1} and all the conflicting TSs for \texttt{app\_0} have
been aggregated so it is also put in the
\emph{AggregatedTransactionState}. Similarly, \texttt{TS\_4} and
\texttt{TS\_5} are also aggregated. Now the
\emph{AggregatedTransactionState} contains the modifications of
\texttt{TS\_1}, \texttt{TS\_2}, \texttt{TS\_3}, \texttt{TS\_4},
\texttt{TS\_5} and begins the commit phase. With the previous commit
mechanism, every TS would have performed the commit phase individually introducing
considerable delay due to the RTT to the database, whereas with the
new commit mechanism we only require two commits in the database.

Aggregating several TSs into a single one introduced some erroneous
behaviour. NDB is very performant with transactions of small size but
the aggregation mechanism was overloading the transaction and NDB was
throwing errors. In order to mitigate this issue a TCP-like aggregation control
was put in place. In the beginning, the mechanism starts to aggregate
a small number of TSs. When the commit phase for that
\emph{AggregatedTransactionState} is successfully completed, the limit
is increased by some delta. It continues to increase until an error
message is received from the database. The transaction is rolled-back,
the limit for the number of aggregations is set to minimum and the
mechanism tries again to aggregate. Users can implement their own
policy as it fits their needs without having to change the commit
mechanism code.

Overall, the new commit mechanism addresses both the waiting time in
the queue and the communication latency. TSs do not have to wait in
the queue for all the conflicting TSs to be committed. Once one
conflicting TS is committed then the mechanism aggregates as many as
possible TSs. Moreover, since multiple TSs are squashed into a single
one, there is only one commit phase reducing the network latency.


\section{Garbage Collector service}
\label{sec:gc_service}
Following the improvements we have made with the Transaction State
commit mechanism we profiled again YARN. We saw that the performance
was in general better but there were still some cases that should be
improved. These include the time spent when committing a Transaction
State in the database. The actual time required to store some
information to the database was too high. Two cases where examined that
they were slow for two different reasons as explained below.

The first set of operations that where slow to commit were regarding
the RPCs we store for recovery. As I have mentioned earlier Hops-YARN
recovery procedure rebuilds the state of the scheduler from the state
persisted in the database. Moreover, every time the
\emph{ApplicationMasterService} or the \emph{ResourceTrackerService}
receive an RPC they persist it in NDB as well. If the RM crashes, it
will restart or the standby RM will become active, read the latest
RPCs stored in the database and simply replay them. RPCs in the
database are stored in the tables that were depicted in Figure
\ref{fig:impl_fk_yarn_rpc}. We can distinguish between two types of
RPCs. The first type is the heartbeat that the
\emph{ApplicationMaster} sends to the \emph{ApplicationMasterService},
in order to report progress of the application, diagnostics and of
course allocation requests. The second type is the heartbeats that
\emph{NodeManager}s send with the statuses of the containers, health
status of the node etc. Unfortunately, these RPCs can get quite big
in size and considering the limitation of a row size in NDB they had
to be split in multiple tables. For example, if a NM runs 1000
containers then in the heartbeat it will include the status for all of
those. Also, when an AM has just started and makes an allocate
request, this request can contain hundreds of requirements. So the
RPCs are split and stored in the tables as illustrated in Figure
\ref{fig:impl_fk_yarn_rpc}. The \texttt{yarn\_appmaster\_rpc} table
holds general information that is common for both RPCs such as the
\texttt{rpcid} and the \texttt{type} of the RPC whereas the
specific information for allocations or NM status is stored in the
other tables.

The life cycle of an RPC is as follows. First a method is invoked by
an RPC, it decodes the request, which is encoded into a Protocol
Buffers encoding. Then it \textbf{stores} the request in NDB into the
respective tables, it gets the current Transaction State where it adds
the \texttt{rpcid}. Events are triggered, containing the
Transaction State and handled by the RM components. When all the
events have been handled the RPCs that were previously stored are deleted
from the database. If the RM crashes somewhere in the middle, then AMs
and NMs do not have to resend the same information, since RM will read
the last heartbeats from the database. Although, that seems a very
good architecture, persisting and deleting thousands of rows in the
database every second is a tough work and a gain of a few milliseconds
in commit time improves the overall performance.

\begin{figure}
\centering
\includegraphics[scale=0.7]{resources/images/Implementation/rpc_fk_overhead.png}
\caption{RPC foreign key constraints micro-benchmark}
\label{fig:impl_fk_overhead}
\end{figure}

If you recall from Section \ref{ssec:impl_fk_appmaster_rpc}, we have
not removed the foreign key constraints from the database schema for
these tables. At that point we discovered that the foreign
key constraints in the RPCs tables really killed performance. I run
some micro-benchmarks to prove the overhead and the results are shown in
Figure \ref{fig:impl_fk_overhead}. The red line in the diagram is the
time needed to remove rows from the tables in Figure
\ref{fig:impl_fk_yarn_rpc}. Then I have changed the schema and removed
the foreign keys. The time to remove the RPCs from the database
without foreign keys is illustrated with the light blue colour. The impact
on performance of foreign keys in our schema was huge. The performance
for persisting data with (blue line) and without (brown) foreign keys
is comparable until 1500 RPCs but then the schema without the foreign
keys performed better. It was clear from the micro-benchmark that
foreign keys should be removed from this set of tables as well. Before
removing completely the constraints we have experimented with a tree
structure of the foreign keys as in Figure
\ref{fig:impl_fk_alternate_schema}. In this schema, only the tables
\texttt{yarn\_heartbeat\_rpc} and \texttt{yarn\_allocate\_rpc} would
have foreign keys to \texttt{yarn\_appmaster\_rpc}. The
micro-benchmark has shown that a schema with two foreign key
constraints has similar performance with a no foreign key constraints
schema. Then the tables specific to AM \emph{allocate} and NM
\emph{heartbeat} would have foreign keys to
\texttt{yarn\_allocate\_rpc} and \texttt{yarn\_heartbeat\_rpc}
respectively. The problem with this setup was the five child tables
of the \texttt{yarn\_allocate\_rpc} table. Considering the overhead of
five foreign keys we have concluded
to remove all the constraints from these tables and implement the
service described in the rest of this section.

\begin{figure}
\centering
\includegraphics[scale=0.3]{resources/images/Implementation/rpc_fk_alternate_schema.png}
\caption{RPC alternative schema}
\label{fig:impl_fk_alternate_schema}
\end{figure}

The removal of the previously persisted RPCs is part of the
transaction that commits scheduling decisions in the database. If we
are slow in this part, this will increment the commit time of the
whole Transaction State which would make other Transaction States to
be blocked more time in the wait queue having a direct impact on the
cluster utilization. Persisting new RPCs is done when an RPC arrives
and does not directly affect the Transaction State commit time. When
the RM recovers from a failure, it goes through the
\texttt{yarn\_appmaster\_rpc} table and reconstructs the stored RPCs.

In the new implementation, we have created an extra table in the
database called \texttt{yarn\_rpc\_gc} which stores the \texttt{rpcid}
and the \texttt{type} of an RPC. The type can be either
\texttt{HEARTBEAT} or \texttt{ALLOCATE}. When an RPC arrives and
invokes a method, this method will get the Transaction State from the
Transaction State Manager. We add the \texttt{rpcid} and the
\texttt{type} of that RPC in the data structures of the TS and are
persisted in the database. When all of the RPCs have been handled
properly, at the same transaction, \textbf{only} the rows from
the \texttt{yarn\_appmaster\_rpc} table are removed and \textbf{not}
from the others, opposed to the previous solution. The entries at the other tables are taken care by the
\emph{Garbage Collector} service.

\emph{Garbage Collector} (GC) is a Hops-YARN service that purges
asynchronously old RPC entries from the database. The main thread of
GC reads the \texttt{rpcid}s and \texttt{type}s from the
\texttt{yarn\_rpc\_gc} table. Then it creates a number of threads that
will delete the rows with a specific \texttt{rpcid} from all the other
tables. The threads will create
queries that will remove the rows where the column \texttt{rpcid}
equals the specific \texttt{rpcid} fetched by the
\texttt{yarn\_rpc\_gc} table. The tables affected for the AMs'
allocate RPCs are:
\begin{itemize}
\item \texttt{yarn\_allocate\_rpc}
\item \texttt{yarn\_allocate\_rpc\_ask}
\item \texttt{yarn\_allocate\_rpc\_blacklist\_add}
\item \texttt{yarn\_allocate\_rpc\_blacklist\_remove}
\item \texttt{yarn\_allocate\_rpc\_release}
\item \texttt{yarn\_allocate\_rpc\_resource\_increase}
\end{itemize}

The rows for the NMs' heartbeats are removed from the tables:
\begin{itemize}
\item \texttt{yarn\_heartbeat\_rpc}
\item \texttt{yarn\_heartbeat\_container\_statuses}
\item \texttt{yarn\_heartbeat\_keepalive\_app}
\end{itemize}

The only performance drawback that we have with the GC service is that
the deletion queries do not operate on primary keys. There is a
trade-off here. The current implementation uses only one table for the
RPCs to be removed that store the \texttt{rpcid} and the
\texttt{type}. For that reason we cannot do primary key operations but
we persist less data in the Transaction State. Since, the removal of
the RPCs is done asynchronously it does not affect the commit time
of the TS. Also, the \texttt{rpcid} in the RPC tables is indexed so we
do not perform a full table scan and is also the partition key so we avoid RTT
among NDB Data Nodes. The other alternative would be to
store all the necessary information to build the primary keys for the
rows to remove. This would effectively mean that we would need to persist more
data during the commit of the TS that might increase the commit
time. Also, we should have had more than one table to store them,
making the schema more complex. Since the Active RM -- scheduler is
already overloaded with making scheduling decisions, handling AM
heartbeats, etc we decided that the Garbage Collector service will run on
the leader of the standby RMs -- ResourceTrackers.

The asynchronous removal of entries from NDB off-loaded the Transaction
State commit phase in that extend that we decided to use it also for
another case that it took more time to persist data than that we
desired. That is the case of Allocate Response. When the AM registers
with the RM or when it requests resources on the cluster, the
scheduler generates an allocate response that is sent back to the
AM. This allocate response is persisted in the database through the
Transaction State for recovery reasons. In some cases it can contain a
lot of information for example when the AM registers and requests all
of its requirements. When the scheduling decision is made, the scheduler creates a response with the
containers requested. The tables used in the database to store such
allocate response are: \texttt{yarn\_allocate\_response},
\texttt{yarn\_allocated\_containers} and
\texttt{yarn\_completed\_containers\_statuses}. The main problem in
that case was that when we were persisting an allocate response, first we
had to delete the previous response generated by the
scheduler. So each addition to the database, implicitly means an extra
deletion. The delete operation should remove probably hundreds of
rows of allocated containers and completed containers' status. We
followed the same method as described above with the RPCs. A new table
was put in the schema, \texttt{yarn\_alloc\_resp\_gc}, that is
storing the necessary information for the GC to remove previous
allocation responses. That way, during the commit phase of the
Transaction State we only add new responses and \textbf{not} removing
the old ones, reducing more the commit time of a TS.

Removing entries from the database in an asynchronous way raises some
discussion about the consistency model. For example what will be the
state recovered by the scheduler if old RPCs or allocate responses
have not been deleted yet. Regarding the recovery of RPCs, when the
failed RM tries to construct the last state from the database, it
fetches all the entries stored in the RPC tables and joins the split
parts. The detail that makes the system work is that it consults the
\texttt{yarn\_appmaster\_rpc} table about the \texttt{rpcid}s of the
RPCs that had not been handled when the RM crashed. Since the entries
for that table are removed synchronously in the commit phase of the
Transaction State, we guarantee that this table will not contain any
rubbish. So the RPCs that will be joined and re-constructed will be
real not handled RPCs. The only drawback is that it might take more
time to read the RPC tables since they might contain more rows than
actually needed. In addition, the GC runs on the RT so until the new
RM makes a transition from standby to active and recover the persisted
state, the old RPCs would have been collected. The same philosophy
applies for the Allocate Responses. The old allocate response of an
application is removed synchronously from table
\texttt{yarn\_allocate\_response}, by the Transaction State. Each
response has an incremented ID. At any time, there is only one ID for
each application attempt stored in this table. When the scheduler
constructs the last allocate response before it crashed, it filters
the entries read from \texttt{yarn\_allocated\_containers} and
\texttt{yarn\_completed\_containers\_statuses} with the valid response ID.

\section{DTO Caching mechanism}
\label{sec:dto_caching}
In Hops in order to communicate with the MySQL Cluster NDB we make use
of ClusterJ \cite{clusterj}, a high level API to perform
operations on NDB. In a sense it is similar to other ORM frameworks
such as Hibernate \cite{hibernate} and EclipseLink \cite{eclipselink}
which provide an object-relational mapping but more lightweight and is
designed to provide high performance methods for storing and accessing
data on a MySQL Cluster from a Java application. Every operation on
Hops and Hops-YARN, except for the events received from the NDB Event
API, goes through ClusterJ which in turn uses the C++ NDB API. The
mapping between a table-oriented view and a Java object is done
through specially decorated interfaces. The interface provides signatures
for the getters and setters methods.

For example Listing \ref{lst:clusterj_intf} shows the interface for
accessing database entries for the Garbage Collector service regarding
old RPCs. The interface is annotated with the table name and contains
signatures for accessing each column of the table. The methods are
also decorated with the primary key annotation and the column
name. For every table in the database there exists such an interface
and all the operations from Hops-YARN are done on the \emph{Data
  Transfer Objects} (DTO) defined
by the annotated interfaces. DTOs are
created from a \emph{session} object which represent a connection to
the MySQL Cluster by calling the \texttt{newInstance} method.

\lstinputlisting[float,language=Java,frame=single,caption={ClusterJ
annotated interface},label=lst:clusterj_intf]{resources/listings/clusterj_intf.java}

Upon completion of the task discussed in the previous section, we
profiled again the commit phase of a Transaction State to discover
spots that could be possibly improved. Surprisingly we discovered that
we suffered from the overhead of creating DTOs with ClusterJ. To
measure the overhead we have created a micro-benchmark that is
creating and persisting a number of DTOs. The results are shown in
Figure \ref{fig:impl_dto_no_cache}. The blue line represents the time
that ClusterJ needed to create the
DTO instances when calling \texttt{session.newInstance}. The yellow
line is the time we spent to persist them in NDB, while
the red one is the sum of those two. Throughout the benchmark we are spending more
time creating the object instances than actually persisting
them. The time to create the DTOs grows linearly to the number of DTOs
and faster than to persist them in the database. Creating more than
6000 DTOs is not an extreme scenario when we aim to scale Hops-YARN over 10000
NodeManagers.

\begin{figure}
\centering
\includegraphics[scale=0.55]{resources/images/Implementation/dto_create_commit_no_cache.png}
\caption{Create and commit time for simple DTOs}
\label{fig:impl_dto_no_cache}
\end{figure}

In ClusterJ, DTOs are created from database \emph{session} object. Creation
of a new DTO involves the instantiation of several objects such as the
handlers for the type of values the DTO will persist in NDB. Upon creation
of all the necessary handlers, it invokes the
\texttt{session.newInstance} reflective method of Java. Java reflection API is
a powerful tool but comes with some pitfalls including
performance \cite{java_reflection}. Reflection API loads types dynamically therefore JVM
optimizations cannot be applied making it a bad candidate for
high-performance applications. Changing the implementation of ClusterJ
is a very difficult task and was not considered as an
option. Moreover, we do not want to maintain one more project.

In Hops, \emph{HopsSession}s are wrapped around ClusterJ sessions. The
solution we designed is a DTO cache for the database sessions.
A session has its own cache space that is filled
up with instances created by the ``slow'' ClusterJ instantiation
process. When we actually need to use a DTO, we fetch it from
the cache which is faster since the objects have already been
created. When the cache has been used a worker thread fills it up
again with new instances. The cache generator service should be used
cautiously. We should not max-out CPUs just for creating cached DTOs, so the
cache is enabled only for a fraction of sessions and only for
heavy-duty DTOs. The reason we decided to have both cache-enabled
and cache-disabled sessions, is to avoid marking a cached session as
used even though the cache was never used. An overall workflow 
of Hops-YARN database session provider
is illustrated in Figure \ref{fig:impl_dto_session_arch}. A
transaction requests a cache-disabled session from the cache-disabled
session pool $(1)$. If there is a session available -- not used by
other transactions -- then the session provider return a session from
the pool, otherwise it creates a new one. When the transaction has
been committed to the database, the session is returned back to the pool
of cache-disabled sessions $(3)$. The workflow for a cache-enabled
session regarding the DB session provider is
different. There are two different cache-enabled session
pools. The first one, \emph{Preparing pool} contains sessions with
their cached used and probably empty. The second pool is the
\emph{Ready pool} with sessions whose cache is full and ready for
usage. When a transaction requests a cache-enabled session, the session
provider first looks on the \emph{Ready pool} $(1)$. If there is an
available session it returns it to the requester $(2)$, otherwise it
looks into the \emph{Preparing pool}. Finally, if there is no session
there either, it creates a new session to NDB. When the transaction
has performed its operations, it returns the session to the
\emph{Preparing pool}. The cache generator service picks sessions from
the \emph{Preparing pool} $(1)$, fills up their cache with the appropriate
DTO instances and places them back to the \emph{Ready pool} $(2)$.

\begin{figure}
\centering
\includegraphics[scale=0.4]{resources/images/Implementation/db_session_pools.png}
\caption{Hops-YARN DB session pool}
\label{fig:impl_dto_session_arch}
\end{figure}

The cache itself, \texttt{DTOCacheImpl} is implemented as a \texttt{ConcurrentHashMap} whose
key is the type of DTO cached and its value is a \texttt{CacheEntry}
object. The \texttt{CacheEntry} is supported by an
\texttt{ArrayBlockingQueue} providing methods for putting and getting
cached objects and increasing the cache size. We will see later how
the size of the cache is increased. For every \texttt{HopsSession}
that has its cache enabled, there is an instance of
\texttt{DTOCacheImpl} which provides methods for (de)registering a DTO
type to the cache and methods that delegate putting and getting objects to the
appropriate \texttt{CacheEntry}. There are
two ways to ``instantiate'' DTO objects. The first one is the
\texttt{newInstance} which makes a call to ClusterJ to create the
object. We use this for non-cached DTO types. The other variation is
the \texttt{newCachedInstance} which makes a call to the cache
instead. In this version the DTO has been instantiated
ahead of time and it is fetched from the cache. The semantics of the
cache is to return the cached object if
it exists in the cache and \texttt{null} if the cache is empty. In
case of an empty cache, we fall back to the ClusterJ instantiation
method. Every \texttt{CacheEntry} keeps track of how many cache-misses
have been occurred. If there are too many, it means that this DTO type
is very demanding and the cache size is increased every time the
cache-misses exceed a certain threshold. So, at the same cache-enabled
session two DTO types might have different cache size depending on
their ``popularity''.

When a cached-enabled session has been used, it is put back in the
\emph{Preparing pool}. The \texttt{DTOCacheGenerator} is a service
that picks sessions from that pool and fills their cache. It removes a
number of sessions from the \emph{Preparing pool} and it spawns threads
that populate every \texttt{CacheEntry} that is not
full. The instantiation of the cached DTO objects is done with the
\texttt{session.newInstance} method of ClusterJ. \texttt{CacheEntry} returns
\texttt{true} for every put, until
the back-end \texttt{ArrayBlockingQueue} is full when it returns
\texttt{false}. When all the \texttt{CacheEntry}s of a session are
full, it (the session) is placed to the \emph{Ready pool} and another
transaction will use it. One note should be made here. ClusterJ
allocates memory for DTO objects out of the heap, with the use of
Java \emph{direct} ByteBuffer. ByteBuffers do not account in Java's
garbage collection mechanism, reducing the GC pause time. Also, they are ideal
for heavy duty I/O operations since JVM does not have to copy data
from intermediate buffers to native buffers. With the caching
mechanism we create more than 6000 DTOs ahead of time per
session and the default direct memory size reaches its limit very
quickly. In order to avoid related exceptions, the flag
\texttt{-XX:MaxDirectMemorySize} should be set to a reasonable value.

\lstinputlisting[float,language=xml,numbers=left,caption={Caching
  mechanism configuration file},label=lst:dto_cache_conf,basicstyle=\footnotesize]{resources/listings/dto-cache-config.xml}

All the parameters for the caching mechanism are specified by a
configuration file, \texttt{dto\_cache-config.xml} in
\texttt{hops-metadata-dal-impl-ndb} that is loaded when
the service starts. A typical example looks like in Listing
\ref{lst:dto_cache_conf}. First, it is specified how many sessions
will have their cache enabled, then a hard limit on the number
of threads that will be created to populate the sessions' cache and
the number of sessions each worker thread will handle. Then in line 14
is the number of sessions in the \emph{Preparing pool} required to
trigger the cache generator threads. If \textbf{sessionsInterval}
$>$ \emph{sessionsPerThread} $*$ \emph{threadLimit}, then some
sessions will be blocked until a thread has finished its work and
scheduled again. Following the mechanism parameters, are the DTO types
that should be cached. In this example we want each cache-enabled
session to cache two types of DTOs. Line 18 is the class name that
encloses the ClusterJ specially decorated interface. Line 21 is the
name of the interface and then follows the initial size of the cache,
the maximum size and the step which the cache size will be increased after some
threshold of cache-misses.

\subsection{Conclusions}
\label{ssec:impl_dto_caching_conslusions}
The caching mechanism explained in this section boosted the
performance of Hops-YARN as it is presented in Chapter \ref{chap:evaluation}. Yet
we have not discovered the advantages in full extend since we need to
experiment more with different types of DTOs. For sure ClusterJ's way
of creating DTOs is not optimal, imposing a great overhead to our system.

\chapter{Analysis}
\label{chap:analysis}
%% How you are going to evaluate what you have done?
%% Analysis of your data and proposed solution
%% Does this meet the goals which you had when you started?
This is going to be the analysis...

\chapter{Conclusions}
\label{chap:conclusion}

\section{Conclusion}
\label{sec:conclusion}
In Hops-YARN, as it is already mentioned, MySQL Cluster NDB is used
both as a communication transport and as a persistent storage for
recovery. At the beginning of the project a thorough profiling of the
execution workflow has been done to identify the bottlenecks of the
system. The first step analyzed in Section \ref{sec:tx_aggregation}
was to improve the commit mechanism of Hops-YARN's Transaction
Manager. The new mechanism squashes several blocked transactions into
a ``big'' one, reducing the number of commits in the back-end database
system. Section \ref{sec:fk_constraints} describes the evolution to a
database schema with no foreign key constraints and how they were
replaced by application logic that performs primary key operations instead.
Section \ref{sec:gc_service} presents the Garbage Collector
service of Hops-YARN that asynchronously removes old values from the
database. With that solution the commit time dropped even more improving
the overall performance of the system. Finally, Section
\ref{sec:dto_caching} explains how the shortcoming of ClusterJ for
creating DTOs was bypassed by having created them ahead of time in a
per session cache.

After a detailed explanation of the solutions proposed, in Chapter
\ref{chap:evaluation} follows the evaluation. Each solution is
evaluated separately by simulating real world traces. In each case key
characteristics are examined and how they have been improved. In
Section \ref{sec:performance_overview} there is an overall performance
overview in terms of cluster utilization and heartbeats processed by
the scheduler. The comparison is made among the version of Hops-YARN
before this project, the final version with all the modification
proposed and the upstream Apache YARN. The figures show that there was
a clear improvement, in both evaluation parameters, when the two
version of Hops-YARN are compared. Finally, as far as the cluster
utilization is concerned, the performance is comparable with Apache
YARN in clusters with thousands of nodes.

\subsection{Goals}
\label{ssec:goals}
\subsection{Goals}
\label{ssec:goals}
That's the goals subsection


\subsection{Insights and suggestions for further work}
\label{ssec:insights-and-suggestions}
This is the insights section

\section{Future work}
\label{sec:future-work}
A few things have been left undone due to time limitation and are
discussed in this section for future work.

In most of the cases the evaluation has been done using simulations
measuring among others the cluster utilization while in others the
evaluation has been done using benchmarks. It would be more complete
if in all cases the benchmarks were supported by simulation
results. That way the perfomance improvements introduced by each step
would be more clear.

Currently we do not have any insight on the content of Transaction
State objects, thus we treat them equally. In real world scenarios,
during the allocation of resources, the Transaction State would carry
more information about allocated containers than when the cluster is
full and no further allocations can be made. A fine-grained inspection
on the content of a TS might improve performance further more. For
example in the commit mechanism, when an Aggregated Transaction State
object is overloading NDB, the transaction will roll-back and the
aggregation policy will enforce a lower limit. If we had any
information on the content of the TS before hand, we could have
avoided that situation.

As it is already mentioned, the Garbage Collector service does not
perform primary key operations. It was a design decision that would
not complicate the mechanism and burden the commit time by committing
more information. It is worth trying to persist all the columns needed
for the primary keys to be reconstructed and measure the
performance. For sure the deletion time would be lower but it
remains to be proven if that will have any impact on the performance
of Hops-YARN in general.

Last but not least, the batching system explained in Section
\ref{ssec:impl_batch_system} should be improved. Changing the number
of RPCs that are batched together did not change the
performance. Every RPC arriving in the ResourceManager among others it
should get the current Transaction State object from the Transaction State
Manager. This operation could take $0.06$ ms. If we consider a cluster
with 10000 NodeManagers heartbeating every second, it is needed 600 ms
just to acquire the object. At the time of writing that was the latest
bottleneck encountered and for sure it is worth for future investigation.

\subsection{What has been left undone?}
\label{ssec:what-has-been-left-undone}
\subsection{What has been left undone?}
\label{what-has-been-left-undone}
This is what has been left undone


%%\bibliography{report}
%%\bibliographystyle{IEEEtran}
%%\bibliographystyle{unsrturl}
%%\bibliographystyle{unsrtnat}
%%\bibliographystyle{myIEEEtran}

\printbibliography[heading=bibintoc]

\appendix
\chapter{Insensible Approximation}

\backmatter

\end{document}
