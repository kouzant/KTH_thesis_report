%%%%%%%%%%%%%%%%%%%%%%%%%%%%%%%%%%%%%%%%%%%%%%%%%%%%%%%%%%%%%%%%%%%%%%%
%% $Id: report.tex,v 1.5 2005/02/09 21:06:42 lindstrm Exp $
%%%%%%%%%%%%%%%%%%%%%%%%%%%%%%%%%%%%%%%%%%%%%%%%%%%%%%%%%%%%%%%%%%%%%%%
%% costhesis usage example
%% modified and added to by GQMJr
%%%%%%%%%%%%%%%%%%%%%%%%%%%%%%%%%%%%%%%%%%%%%%%%%%%%%%%%%%%%%%%%%%%%%%%
%
% The costhesis package accepts the following options
%
%   Document types:
%     msc               - Master Thesis
%     bsc		- Kandidate Thesis
%
%   Layout options:
%
%   Other options:
%     blank             - Removes pagenumbers and headers from empty pages
%     blankmsg          - Prints a message of intent on empty pages
%     scheader          - Typeset headers in SMALL CAPS shape (default)
%     slheader          - Typeset headers in slanted shape 
%
%
%
%

\documentclass[12pt,a4paper,twoside,openright]{book}
%%\documentclass[12pt,a4paper,twoside,openright]{memoir}

\usepackage[msc,blankmsg]{costhesis}
%\usepackage[T1]{fontenc}
%%\usepackage{pslatex}
\renewcommand{\rmdefault}{ptm} 
\usepackage{mathptmx}
\usepackage[scaled=.90]{helvet}
\usepackage{courier}
%
\usepackage{bookmark}


%%----------------------------------------------------------------------------
%%   pcap2tex stuff
%%----------------------------------------------------------------------------
 \usepackage[dvipsnames*,svgnames]{xcolor} %% For extended colors
 \usepackage{tikz}
 \usetikzlibrary{arrows,decorations.pathmorphing,backgrounds,fit,positioning,calc,shapes}
 \usepackage{pgfmath}	% --math engine
%%----------------------------------------------------------------------------
%\usepackage[latin1]{inputenc}
\usepackage[utf8]{inputenc} % inputenc allows the user to input accented characters directly from the keyboard
\usepackage[swedish,english]{babel}
\usepackage{rotating}		 %% For text rotating
\usepackage{array}			 %% For table wrapping
\usepackage{graphicx}	 %% Support for images
\usepackage{float}			 %% Suppor for more flexible floating box positioning
\usepackage{color}           %% Support for colour 
\usepackage{mdwlist}
\usepackage{setspace}    %% For fine-grained control over line spacing
\usepackage{listings}		%% For source code listing
\usepackage{bytefield}    %% For packet drawings
\usepackage{tabularx}		%% For simple table stretching
\usepackage{multirow}	%% Support for multirow colums in tables
\usepackage{dcolumn}	%% Support for decimal point alignment in tables
\usepackage{url}	%% Support for breaking URLs
\usepackage[perpage,para,symbol]{footmisc} %% use symbols to ``number'' footnotes and reset which symbol is used first on each page

%%\usepackage{pygmentize}  %% required to use minted -- see python-pygments - Pygments is a Syntax Highlighting Package written in Python
%%\usepackage{minted}		%% For source code highlighting

\usepackage{hyperref}		
\usepackage[all]{hypcap}	 %% Prevents an issue related to hyperref and caption linking
%% setup hyperref to use the darkblue color on links
\hypersetup{colorlinks,breaklinks,
            linkcolor=darkblue,urlcolor=darkblue,
            anchorcolor=darkblue,citecolor=darkblue}


%% Some definitions of used colors
\definecolor{darkblue}{rgb}{0.0,0.0,0.3} %% define a color called darkblue
\definecolor{darkred}{rgb}{0.4,0.0,0.0}
\definecolor{red}{rgb}{0.7,0.0,0.0}
\definecolor{lightgrey}{rgb}{0.8,0.8,0.8} 
\definecolor{grey}{rgb}{0.6,0.6,0.6}
\definecolor{darkgrey}{rgb}{0.4,0.4,0.4}
%% Reduce hyphenation as much as possible
\hyphenpenalty=15000 
\tolerance=1000

%% useful redefinitions to use with tables
\newcommand{\rr}{\raggedright} %% raggedright command redefinition
\newcommand{\rl}{\raggedleft} %% raggedleft command redefinition
\newcommand{\tn}{\tabularnewline} %% tabularnewline command redefinition

%% definition of new command for bytefield package
\newcommand{\colorbitbox}[3]{%
	\rlap{\bitbox{#2}{\color{#1}\rule{\width}{\height}}}%
	\bitbox{#2}{#3}}

%% command to ease switching to red color text
\newcommand{\red}{\color{red}}
%%redefinition of paragraph command to insert a breakline after it
\makeatletter
\renewcommand\paragraph{\@startsection{paragraph}{4}{\z@}%
  {-3.25ex\@plus -1ex \@minus -.2ex}%
  {1.5ex \@plus .2ex}%
  {\normalfont\normalsize\bfseries}}
\makeatother

%%redefinition of subparagraph command to insert a breakline after it
\makeatletter
\renewcommand\subparagraph{\@startsection{subparagraph}{5}{\z@}%
  {-3.25ex\@plus -1ex \@minus -.2ex}%
  {1.5ex \@plus .2ex}%
  {\normalfont\normalsize\bfseries}}
\makeatother

\setcounter{tocdepth}{3}	%% 3 depth levels in TOC
\setcounter{secnumdepth}{5} %% 3 sectioning levels. WARNING: command \mainmatter resets this field to its default value!!!
%%%%%%%%%%%%%%%%%%%%%%%%%%%%%%%%%%%%%%%%%%%%%%%%%%%%%%%%%%%%%%%%%%%%
%% End of preamble
%%%%%%%%%%%%%%%%%%%%%%%%%%%%%%%%%%%%%%%%%%%%%%%%%%%%%%%%%%%%%%%%%%%%

\iauthor{My Name}
\ititle{My Report Title}
\isubtitle{My Report Subtitle}
\idate{2012}{January}{21}
\examinername{Professor X}

\setlength{\headheight}{15pt}
\begin{document}

\frontmatter
\selectlanguage{english}
\begin{abstract}
\label{sec:abstract}
\setcounter{page}{1}


Your abstract here.

\end{abstract}
%%\clearpage
\selectlanguage{swedish}
%%\chapter*{Sammanfattning}
\begin{abstract}
\label{sec:swedish_abstract}


IETF xxxx Arbetsgruppen har definierat 
\end{abstract}

\selectlanguage{english}
\begin{acknowledgements}
I would like to acknowldge my adviser's help in getting access to the
necessary packet traffic at a commercial operator (who should be thanked but
must remain unnamed).
\end{acknowledgements}

\selectlanguage{english}
\tableofcontents

\listoffigures

\listoftables

%% add a list of listing if and listings are used
%%\listoflistings

% \begin{notations}
% \end{notations}

\renewcommand\abbreviationsname{List of Acronyms and Abbreviations}
\begin{abbreviations}
\label{list-of-acronyms-and-abbreviations}

This document requires readers to be familiar with terms and concepts described in \mbox{RFC~1235} \cite{john_ioannidis_coherent_1991}. For clarity we summarize some of these terms and give a short description of them before presenting them in next sections.

\begin{basedescript}{\desclabelstyle{\pushlabel}\desclabelwidth{10em}}
\item[IPv4]					Internet Protocol version 4 (RFC~791 \cite{postel_internet_1981})
\item[IPv6]					Internet Protocol version 6 (RFC~2460 \cite{deering_internet_1998})
\end{basedescript}
\end{abbreviations}

\mainmatter
\setcounter{secnumdepth}{5} 
\chapter{Introduction}
\label{chap:introduction}
%% Longer problem statement
%% General introduction to the area

\section{Problem description}
\label{sec:problem_description}


\section{Problem context}
\label{sec:problem_context}


\section{Structure of this thesis}
\label{sec:thesis_structure}


\chapter{Background}
\label{chap:background}

%%    What does a reader (another x student -- where x is your study line) need to know to understand your report?
%%    What have others already done?

\chapter{Method}
\label{chap:method}

%% What are your goals? (What should you be able to do as a result of your solution - which couldn't be done well before you started?)
%%  What you are going to do? Why?

\chapter{Analysis}
\label{chap:analysis}

%% How you are going to evaluate what you have done?
%% Analysis of your data and proposed solution
%% Does this meet the goals which you had when you started?


\chapter{Conclusions}
\label{chap:conclusion}

\section{Conclusion}
%% Did you meet your goals?
%% What insights have you gained?
%% What suggestions can you give to others working in this area?
%% If you had it to do again, what would you have done differently?


In this section we will state the conclusions and insights gained as result of this thesis project.

\subsection{Goals}
\label{ssec:goals}


\subsection{Insights and suggestions for further work}
\label{ssec:insights-and-suggestions}

\section{Future work}
\label{sec:future-work}
%% What you have left undone?
%% What are the next obvious things to be done?
%% What hints can you give to the next person who is going to followup upon your work?

\subsection{What has been left undone?}
\label{what-has-been-left-undone}


\bibliography{report}
%%\bibliographystyle{IEEEtran}
%%\bibliographystyle{unsrturl}
%%\bibliographystyle{unsrtnat}
\bibliographystyle{myIEEEtran}
\appendix
\chapter{Insensible Approximation}

\backmatter

\end{document}
